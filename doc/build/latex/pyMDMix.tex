% Generated by Sphinx.
\def\sphinxdocclass{report}
\documentclass[letterpaper,10pt,english]{sphinxmanual}
\usepackage[utf8]{inputenc}
\DeclareUnicodeCharacter{00A0}{\nobreakspace}
\usepackage{cmap}
\usepackage[T1]{fontenc}
\usepackage{babel}
\usepackage{times}
\usepackage[Bjarne]{fncychap}
\usepackage{longtable}
\usepackage{sphinx}
\usepackage{multirow}


\title{pyMDMix Documentation}
\date{March 11, 2014}
\release{0.1.1}
\author{Daniel Alvarez-Garcia}
\newcommand{\sphinxlogo}{}
\renewcommand{\releasename}{Release}
\makeindex

\makeatletter
\def\PYG@reset{\let\PYG@it=\relax \let\PYG@bf=\relax%
    \let\PYG@ul=\relax \let\PYG@tc=\relax%
    \let\PYG@bc=\relax \let\PYG@ff=\relax}
\def\PYG@tok#1{\csname PYG@tok@#1\endcsname}
\def\PYG@toks#1+{\ifx\relax#1\empty\else%
    \PYG@tok{#1}\expandafter\PYG@toks\fi}
\def\PYG@do#1{\PYG@bc{\PYG@tc{\PYG@ul{%
    \PYG@it{\PYG@bf{\PYG@ff{#1}}}}}}}
\def\PYG#1#2{\PYG@reset\PYG@toks#1+\relax+\PYG@do{#2}}

\expandafter\def\csname PYG@tok@gd\endcsname{\def\PYG@tc##1{\textcolor[rgb]{0.63,0.00,0.00}{##1}}}
\expandafter\def\csname PYG@tok@gu\endcsname{\let\PYG@bf=\textbf\def\PYG@tc##1{\textcolor[rgb]{0.50,0.00,0.50}{##1}}}
\expandafter\def\csname PYG@tok@gt\endcsname{\def\PYG@tc##1{\textcolor[rgb]{0.00,0.27,0.87}{##1}}}
\expandafter\def\csname PYG@tok@gs\endcsname{\let\PYG@bf=\textbf}
\expandafter\def\csname PYG@tok@gr\endcsname{\def\PYG@tc##1{\textcolor[rgb]{1.00,0.00,0.00}{##1}}}
\expandafter\def\csname PYG@tok@cm\endcsname{\let\PYG@it=\textit\def\PYG@tc##1{\textcolor[rgb]{0.25,0.50,0.56}{##1}}}
\expandafter\def\csname PYG@tok@vg\endcsname{\def\PYG@tc##1{\textcolor[rgb]{0.73,0.38,0.84}{##1}}}
\expandafter\def\csname PYG@tok@m\endcsname{\def\PYG@tc##1{\textcolor[rgb]{0.13,0.50,0.31}{##1}}}
\expandafter\def\csname PYG@tok@mh\endcsname{\def\PYG@tc##1{\textcolor[rgb]{0.13,0.50,0.31}{##1}}}
\expandafter\def\csname PYG@tok@cs\endcsname{\def\PYG@tc##1{\textcolor[rgb]{0.25,0.50,0.56}{##1}}\def\PYG@bc##1{\setlength{\fboxsep}{0pt}\colorbox[rgb]{1.00,0.94,0.94}{\strut ##1}}}
\expandafter\def\csname PYG@tok@ge\endcsname{\let\PYG@it=\textit}
\expandafter\def\csname PYG@tok@vc\endcsname{\def\PYG@tc##1{\textcolor[rgb]{0.73,0.38,0.84}{##1}}}
\expandafter\def\csname PYG@tok@il\endcsname{\def\PYG@tc##1{\textcolor[rgb]{0.13,0.50,0.31}{##1}}}
\expandafter\def\csname PYG@tok@go\endcsname{\def\PYG@tc##1{\textcolor[rgb]{0.20,0.20,0.20}{##1}}}
\expandafter\def\csname PYG@tok@cp\endcsname{\def\PYG@tc##1{\textcolor[rgb]{0.00,0.44,0.13}{##1}}}
\expandafter\def\csname PYG@tok@gi\endcsname{\def\PYG@tc##1{\textcolor[rgb]{0.00,0.63,0.00}{##1}}}
\expandafter\def\csname PYG@tok@gh\endcsname{\let\PYG@bf=\textbf\def\PYG@tc##1{\textcolor[rgb]{0.00,0.00,0.50}{##1}}}
\expandafter\def\csname PYG@tok@ni\endcsname{\let\PYG@bf=\textbf\def\PYG@tc##1{\textcolor[rgb]{0.84,0.33,0.22}{##1}}}
\expandafter\def\csname PYG@tok@nl\endcsname{\let\PYG@bf=\textbf\def\PYG@tc##1{\textcolor[rgb]{0.00,0.13,0.44}{##1}}}
\expandafter\def\csname PYG@tok@nn\endcsname{\let\PYG@bf=\textbf\def\PYG@tc##1{\textcolor[rgb]{0.05,0.52,0.71}{##1}}}
\expandafter\def\csname PYG@tok@no\endcsname{\def\PYG@tc##1{\textcolor[rgb]{0.38,0.68,0.84}{##1}}}
\expandafter\def\csname PYG@tok@na\endcsname{\def\PYG@tc##1{\textcolor[rgb]{0.25,0.44,0.63}{##1}}}
\expandafter\def\csname PYG@tok@nb\endcsname{\def\PYG@tc##1{\textcolor[rgb]{0.00,0.44,0.13}{##1}}}
\expandafter\def\csname PYG@tok@nc\endcsname{\let\PYG@bf=\textbf\def\PYG@tc##1{\textcolor[rgb]{0.05,0.52,0.71}{##1}}}
\expandafter\def\csname PYG@tok@nd\endcsname{\let\PYG@bf=\textbf\def\PYG@tc##1{\textcolor[rgb]{0.33,0.33,0.33}{##1}}}
\expandafter\def\csname PYG@tok@ne\endcsname{\def\PYG@tc##1{\textcolor[rgb]{0.00,0.44,0.13}{##1}}}
\expandafter\def\csname PYG@tok@nf\endcsname{\def\PYG@tc##1{\textcolor[rgb]{0.02,0.16,0.49}{##1}}}
\expandafter\def\csname PYG@tok@si\endcsname{\let\PYG@it=\textit\def\PYG@tc##1{\textcolor[rgb]{0.44,0.63,0.82}{##1}}}
\expandafter\def\csname PYG@tok@s2\endcsname{\def\PYG@tc##1{\textcolor[rgb]{0.25,0.44,0.63}{##1}}}
\expandafter\def\csname PYG@tok@vi\endcsname{\def\PYG@tc##1{\textcolor[rgb]{0.73,0.38,0.84}{##1}}}
\expandafter\def\csname PYG@tok@nt\endcsname{\let\PYG@bf=\textbf\def\PYG@tc##1{\textcolor[rgb]{0.02,0.16,0.45}{##1}}}
\expandafter\def\csname PYG@tok@nv\endcsname{\def\PYG@tc##1{\textcolor[rgb]{0.73,0.38,0.84}{##1}}}
\expandafter\def\csname PYG@tok@s1\endcsname{\def\PYG@tc##1{\textcolor[rgb]{0.25,0.44,0.63}{##1}}}
\expandafter\def\csname PYG@tok@gp\endcsname{\let\PYG@bf=\textbf\def\PYG@tc##1{\textcolor[rgb]{0.78,0.36,0.04}{##1}}}
\expandafter\def\csname PYG@tok@sh\endcsname{\def\PYG@tc##1{\textcolor[rgb]{0.25,0.44,0.63}{##1}}}
\expandafter\def\csname PYG@tok@ow\endcsname{\let\PYG@bf=\textbf\def\PYG@tc##1{\textcolor[rgb]{0.00,0.44,0.13}{##1}}}
\expandafter\def\csname PYG@tok@sx\endcsname{\def\PYG@tc##1{\textcolor[rgb]{0.78,0.36,0.04}{##1}}}
\expandafter\def\csname PYG@tok@bp\endcsname{\def\PYG@tc##1{\textcolor[rgb]{0.00,0.44,0.13}{##1}}}
\expandafter\def\csname PYG@tok@c1\endcsname{\let\PYG@it=\textit\def\PYG@tc##1{\textcolor[rgb]{0.25,0.50,0.56}{##1}}}
\expandafter\def\csname PYG@tok@kc\endcsname{\let\PYG@bf=\textbf\def\PYG@tc##1{\textcolor[rgb]{0.00,0.44,0.13}{##1}}}
\expandafter\def\csname PYG@tok@c\endcsname{\let\PYG@it=\textit\def\PYG@tc##1{\textcolor[rgb]{0.25,0.50,0.56}{##1}}}
\expandafter\def\csname PYG@tok@mf\endcsname{\def\PYG@tc##1{\textcolor[rgb]{0.13,0.50,0.31}{##1}}}
\expandafter\def\csname PYG@tok@err\endcsname{\def\PYG@bc##1{\setlength{\fboxsep}{0pt}\fcolorbox[rgb]{1.00,0.00,0.00}{1,1,1}{\strut ##1}}}
\expandafter\def\csname PYG@tok@kd\endcsname{\let\PYG@bf=\textbf\def\PYG@tc##1{\textcolor[rgb]{0.00,0.44,0.13}{##1}}}
\expandafter\def\csname PYG@tok@ss\endcsname{\def\PYG@tc##1{\textcolor[rgb]{0.32,0.47,0.09}{##1}}}
\expandafter\def\csname PYG@tok@sr\endcsname{\def\PYG@tc##1{\textcolor[rgb]{0.14,0.33,0.53}{##1}}}
\expandafter\def\csname PYG@tok@mo\endcsname{\def\PYG@tc##1{\textcolor[rgb]{0.13,0.50,0.31}{##1}}}
\expandafter\def\csname PYG@tok@mi\endcsname{\def\PYG@tc##1{\textcolor[rgb]{0.13,0.50,0.31}{##1}}}
\expandafter\def\csname PYG@tok@kn\endcsname{\let\PYG@bf=\textbf\def\PYG@tc##1{\textcolor[rgb]{0.00,0.44,0.13}{##1}}}
\expandafter\def\csname PYG@tok@o\endcsname{\def\PYG@tc##1{\textcolor[rgb]{0.40,0.40,0.40}{##1}}}
\expandafter\def\csname PYG@tok@kr\endcsname{\let\PYG@bf=\textbf\def\PYG@tc##1{\textcolor[rgb]{0.00,0.44,0.13}{##1}}}
\expandafter\def\csname PYG@tok@s\endcsname{\def\PYG@tc##1{\textcolor[rgb]{0.25,0.44,0.63}{##1}}}
\expandafter\def\csname PYG@tok@kp\endcsname{\def\PYG@tc##1{\textcolor[rgb]{0.00,0.44,0.13}{##1}}}
\expandafter\def\csname PYG@tok@w\endcsname{\def\PYG@tc##1{\textcolor[rgb]{0.73,0.73,0.73}{##1}}}
\expandafter\def\csname PYG@tok@kt\endcsname{\def\PYG@tc##1{\textcolor[rgb]{0.56,0.13,0.00}{##1}}}
\expandafter\def\csname PYG@tok@sc\endcsname{\def\PYG@tc##1{\textcolor[rgb]{0.25,0.44,0.63}{##1}}}
\expandafter\def\csname PYG@tok@sb\endcsname{\def\PYG@tc##1{\textcolor[rgb]{0.25,0.44,0.63}{##1}}}
\expandafter\def\csname PYG@tok@k\endcsname{\let\PYG@bf=\textbf\def\PYG@tc##1{\textcolor[rgb]{0.00,0.44,0.13}{##1}}}
\expandafter\def\csname PYG@tok@se\endcsname{\let\PYG@bf=\textbf\def\PYG@tc##1{\textcolor[rgb]{0.25,0.44,0.63}{##1}}}
\expandafter\def\csname PYG@tok@sd\endcsname{\let\PYG@it=\textit\def\PYG@tc##1{\textcolor[rgb]{0.25,0.44,0.63}{##1}}}

\def\PYGZbs{\char`\\}
\def\PYGZus{\char`\_}
\def\PYGZob{\char`\{}
\def\PYGZcb{\char`\}}
\def\PYGZca{\char`\^}
\def\PYGZam{\char`\&}
\def\PYGZlt{\char`\<}
\def\PYGZgt{\char`\>}
\def\PYGZsh{\char`\#}
\def\PYGZpc{\char`\%}
\def\PYGZdl{\char`\$}
\def\PYGZhy{\char`\-}
\def\PYGZsq{\char`\'}
\def\PYGZdq{\char`\"}
\def\PYGZti{\char`\~}
% for compatibility with earlier versions
\def\PYGZat{@}
\def\PYGZlb{[}
\def\PYGZrb{]}
\makeatother

\begin{document}

\maketitle
\tableofcontents
\phantomsection\label{index::doc}


Contents:


\chapter{Project Module documentation}
\label{project:module-pyMDMix.Project}\label{project:welcome-to-pymdmix-s-documentation}\label{project::doc}\label{project:project-module-documentation}\index{pyMDMix.Project (module)}

\section{Project class}
\label{project:project-class}\index{Project (class in pyMDMix.Project)}

\begin{fulllineitems}
\phantomsection\label{project:pyMDMix.Project.Project}\pysiglinewithargsret{\strong{class }\code{pyMDMix.Project.}\bfcode{Project}}{\emph{name='mdmix\_project'}, \emph{amberPDB=None}, \emph{amberOFF=None}, \emph{projectFilePath=None}, \emph{unitName=None}, \emph{extraResList=}\optional{}, \emph{restrainMask='auto'}, \emph{alignMask='auto'}, \emph{extraFF=}\optional{}, \emph{**kwargs}}{}
Class to hold all project related options
\index{\_\_init\_\_() (pyMDMix.Project.Project method)}

\begin{fulllineitems}
\phantomsection\label{project:pyMDMix.Project.Project.__init__}\pysiglinewithargsret{\bfcode{\_\_init\_\_}}{\emph{name='mdmix\_project'}, \emph{amberPDB=None}, \emph{amberOFF=None}, \emph{projectFilePath=None}, \emph{unitName=None}, \emph{extraResList=}\optional{}, \emph{restrainMask='auto'}, \emph{alignMask='auto'}, \emph{extraFF=}\optional{}, \emph{**kwargs}}{}~\begin{description}
\item[{A project can be initialized in \textbf{four} ways::}] \leavevmode\begin{itemize}
\item {} 
Empty: The user may call later {\hyperref[project:pyMDMix.Project.Project.setOFF]{\code{setOFF()}}} or {\hyperref[project:pyMDMix.Project.Project.createOFFFromPDB]{\code{createOFFFromPDB()}}} to set expected Amber object file and system unit name needed by the project to proceed.

\item {} 
From existing project file: \emph{projectFilePath} can point to a folder containing a project file (*.mproj) to be loaded. All information will be loaded from existing sources.

\item {} 
With a PDB file: \emph{amberPDB} can point to a PDB file path with a system ready to be converted to Object File (main input to Project). If the PDB contains non-standard residues, the user can provide forcefield modification files or parameters in \emph{extraFF} list.

\item {} 
From an Amber Object File (OFF): Pass in \emph{amberOFF} a path to an amber object file containing the system to simulate. Extra forcefield parameters and modifications can be passed in \emph{extraFF} list.

\end{itemize}

\end{description}

\end{fulllineitems}

\index{addNewReplica() (pyMDMix.Project.Project method)}

\begin{fulllineitems}
\phantomsection\label{project:pyMDMix.Project.Project.addNewReplica}\pysiglinewithargsret{\bfcode{addNewReplica}}{\emph{replica}, \emph{updateReplica=True}}{}
Add new replica to current project. If replica.name exists in current project,
a warning will be raised and nothing done.

If the new replica does not have a name, an automatic one will be assigned with format
SOLVENT\_NUM depending on the number of replicas per solvent already present in current project
(e.g. ETA\_0 if its the first replica with ETA as solvent; MAM\_3 if it's the fourth replica with this
same solvent).

If \emph{updateReplica} is \textbf{True}, replica restrain mask, align mask, forcefield information
extra residues and reference PDB will be replaced with current project information.
\begin{quote}\begin{description}
\item[{Parameters}] \leavevmode\begin{itemize}
\item {} 
\textbf{replica} ({\hyperref[replicas:pyMDMix.Replicas.Replica]{\code{Replica}}}) -- Replica to add to current project. It can be created later.

\item {} 
\textbf{updateReplica} (\emph{bool}) -- Update replica information with project details.

\end{itemize}

\end{description}\end{quote}

\end{fulllineitems}

\index{amberOFF (pyMDMix.Project.Project attribute)}

\begin{fulllineitems}
\phantomsection\label{project:pyMDMix.Project.Project.amberOFF}\pysigline{\bfcode{amberOFF}\strong{ = None}}
Amber object file path with the system to be studied

\end{fulllineitems}

\index{amberPDB (pyMDMix.Project.Project attribute)}

\begin{fulllineitems}
\phantomsection\label{project:pyMDMix.Project.Project.amberPDB}\pysigline{\bfcode{amberPDB}\strong{ = None}}
PDB file to be used in the preparation of an Amber Object File

\end{fulllineitems}

\index{createGroup() (pyMDMix.Project.Project method)}

\begin{fulllineitems}
\phantomsection\label{project:pyMDMix.Project.Project.createGroup}\pysiglinewithargsret{\bfcode{createGroup}}{\emph{groupname}, \emph{replicanames}}{}
Create a group of replicas for joint analysis
\begin{quote}\begin{description}
\item[{Parameters}] \leavevmode\begin{itemize}
\item {} 
\textbf{groupname} (\emph{str}) -- Name to identify the group

\item {} 
\textbf{replicanames} (\emph{list}) -- Replica names to add to group

\end{itemize}

\end{description}\end{quote}

\end{fulllineitems}

\index{createNewReplicas() (pyMDMix.Project.Project method)}

\begin{fulllineitems}
\phantomsection\label{project:pyMDMix.Project.Project.createNewReplicas}\pysiglinewithargsret{\bfcode{createNewReplicas}}{}{}
Create folder structure and MD input for replicas not already created.
If the project folder is not created, it will also be created.

\end{fulllineitems}

\index{createOFFFromPDB() (pyMDMix.Project.Project method)}

\begin{fulllineitems}
\phantomsection\label{project:pyMDMix.Project.Project.createOFFFromPDB}\pysiglinewithargsret{\bfcode{createOFFFromPDB}}{\emph{amberPDB}, \emph{**kwargs}}{}
Create an amber object file from the pdb given.
This PDB file should be prepared to be read by tLeap (correct residue and atom namings)
Will automatically try to cap the pdb at all terminus and build SS bonds.

This method will save and return the object file and the unit name containing the system.
Call {\hyperref[project:pyMDMix.Project.Project.setOFF]{\code{setOFF()}}} to assign them to the project.
\begin{quote}\begin{description}
\item[{Parameters}] \leavevmode
\textbf{amberPDB} (\emph{str}) -- File path to PDB to be saved as object file

\item[{Kwargs}] \leavevmode
Check \code{cleanPDB()} method for extra parameters that are accepted to control automatic cleaning of the PDB.

\item[{Returns}] \leavevmode
Object file path recently created and the unit name containing the system.

\item[{Return type}] \leavevmode
(str, str)

\end{description}\end{quote}

\end{fulllineitems}

\index{createProject() (pyMDMix.Project.Project method)}

\begin{fulllineitems}
\phantomsection\label{project:pyMDMix.Project.Project.createProject}\pysiglinewithargsret{\bfcode{createProject}}{}{}
Write directory tree and project files. This method will call createProjectFolder and createProjectFiles.

\end{fulllineitems}

\index{createProjectFiles() (pyMDMix.Project.Project method)}

\begin{fulllineitems}
\phantomsection\label{project:pyMDMix.Project.Project.createProjectFiles}\pysiglinewithargsret{\bfcode{createProjectFiles}}{}{}
Save reference PDB, PRMTOP and PRMCRD from Project Object File.

\end{fulllineitems}

\index{createProjectFolder() (pyMDMix.Project.Project method)}

\begin{fulllineitems}
\phantomsection\label{project:pyMDMix.Project.Project.createProjectFolder}\pysiglinewithargsret{\bfcode{createProjectFolder}}{}{}
Create folder structure for current project name and update paths

\end{fulllineitems}

\index{extraFF (pyMDMix.Project.Project attribute)}

\begin{fulllineitems}
\phantomsection\label{project:pyMDMix.Project.Project.extraFF}\pysigline{\bfcode{extraFF}\strong{ = None}}
Forfecield parameters or frcmods to define the system

\end{fulllineitems}

\index{getGroup() (pyMDMix.Project.Project method)}

\begin{fulllineitems}
\phantomsection\label{project:pyMDMix.Project.Project.getGroup}\pysiglinewithargsret{\bfcode{getGroup}}{\emph{groupname}}{}
Get a list with replicas belonging to the group \emph{groupname}
\begin{quote}\begin{description}
\item[{Parameters}] \leavevmode
\textbf{groupname} (\emph{str}) -- Name of the group to retrieve

\item[{Returns}] \leavevmode
List with {\hyperref[replicas:pyMDMix.Replicas.Replica]{\code{Replica}}} instances or False if group does not exists

\end{description}\end{quote}

\end{fulllineitems}

\index{getSolventList() (pyMDMix.Project.Project method)}

\begin{fulllineitems}
\phantomsection\label{project:pyMDMix.Project.Project.getSolventList}\pysiglinewithargsret{\bfcode{getSolventList}}{}{}
Return list of solvents used in current project

\end{fulllineitems}

\index{load() (pyMDMix.Project.Project method)}

\begin{fulllineitems}
\phantomsection\label{project:pyMDMix.Project.Project.load}\pysiglinewithargsret{\bfcode{load}}{\emph{projfile=None}}{}
Load existing project from pickled file

\end{fulllineitems}

\index{projFileName (pyMDMix.Project.Project attribute)}

\begin{fulllineitems}
\phantomsection\label{project:pyMDMix.Project.Project.projFileName}\pysigline{\bfcode{projFileName}\strong{ = None}}
Name of the file that will be generated to save all project information

\end{fulllineitems}

\index{projName (pyMDMix.Project.Project attribute)}

\begin{fulllineitems}
\phantomsection\label{project:pyMDMix.Project.Project.projName}\pysigline{\bfcode{projName}\strong{ = None}}
Name of the project

\end{fulllineitems}

\index{removeGroup() (pyMDMix.Project.Project method)}

\begin{fulllineitems}
\phantomsection\label{project:pyMDMix.Project.Project.removeGroup}\pysiglinewithargsret{\bfcode{removeGroup}}{\emph{groupname}}{}
Remove group.
:arg str groupname: group name to be removed from project

\end{fulllineitems}

\index{removeReplica() (pyMDMix.Project.Project method)}

\begin{fulllineitems}
\phantomsection\label{project:pyMDMix.Project.Project.removeReplica}\pysiglinewithargsret{\bfcode{removeReplica}}{\emph{replicaname}}{}
Remove replica from current project
\begin{quote}\begin{description}
\item[{Parameters}] \leavevmode
\textbf{replica} (\emph{str}) -- replica name to be removed from project

\item[{Returns}] \leavevmode
\emph{True} if replica was removed. \emph{False} if the name was not found.

\end{description}\end{quote}

\end{fulllineitems}

\index{setOFF() (pyMDMix.Project.Project method)}

\begin{fulllineitems}
\phantomsection\label{project:pyMDMix.Project.Project.setOFF}\pysiglinewithargsret{\bfcode{setOFF}}{\emph{offpath}, \emph{unitname=False}}{}
Set current project amber object file containing the system under study.
\begin{quote}\begin{description}
\item[{Parameters}] \leavevmode\begin{itemize}
\item {} 
\textbf{offpath} (\emph{str}) -- Valid file path to an amber object file.

\item {} 
\textbf{unitname} (\emph{str}) -- Unit name inside the file that contains the system. If not given, will automatically use the first unit found in the file.

\end{itemize}

\end{description}\end{quote}

\end{fulllineitems}

\index{unitName (pyMDMix.Project.Project attribute)}

\begin{fulllineitems}
\phantomsection\label{project:pyMDMix.Project.Project.unitName}\pysigline{\bfcode{unitName}\strong{ = None}}
Name of the unit inside the AmberOFF containing the system in study

\end{fulllineitems}

\index{updatePath() (pyMDMix.Project.Project method)}

\begin{fulllineitems}
\phantomsection\label{project:pyMDMix.Project.Project.updatePath}\pysiglinewithargsret{\bfcode{updatePath}}{}{}
Update main project folder path with current working directory.
All project replica's paths will be also updated.

This method to work requires that the expected project file ({\hyperref[project:pyMDMix.Project.Project.projFileName]{\code{Project.projFileName}}}) is placed inside the current working directory.

\end{fulllineitems}

\index{updateReplica() (pyMDMix.Project.Project method)}

\begin{fulllineitems}
\phantomsection\label{project:pyMDMix.Project.Project.updateReplica}\pysiglinewithargsret{\bfcode{updateReplica}}{\emph{replica}}{}
Update replica. When a replica is modified, it is recomended to run this method to update project file information.
:arg replica: replica instance to be updated in current project
:type replica: {\hyperref[replicas:pyMDMix.Replicas.Replica]{\code{Replica}}}

\end{fulllineitems}

\index{updateReplicaPaths() (pyMDMix.Project.Project method)}

\begin{fulllineitems}
\phantomsection\label{project:pyMDMix.Project.Project.updateReplicaPaths}\pysiglinewithargsret{\bfcode{updateReplicaPaths}}{}{}
Update replica paths with current project main path information

\end{fulllineitems}

\index{write() (pyMDMix.Project.Project method)}

\begin{fulllineitems}
\phantomsection\label{project:pyMDMix.Project.Project.write}\pysiglinewithargsret{\bfcode{write}}{}{}
Save object \_\_dict\_\_ to pickled file.

\end{fulllineitems}


\end{fulllineitems}



\chapter{Replicas Module documentation}
\label{replicas:replicas-module-documentation}\label{replicas:module-pyMDMix.Replicas}\label{replicas::doc}\index{pyMDMix.Replicas (module)}
This module contains the main class {\hyperref[replicas:pyMDMix.Replicas.Replica]{\code{Replica}}} for storage and manipulation
of a single simulation run.


\section{Aid to developers}
\label{replicas:aid-to-developers}

\subsection{Replica flexible configuration}
\label{replicas:replica-flexible-configuration}
{\hyperref[replicas:pyMDMix.Replicas.Replica]{\code{Replica}}} objects can be configured by mean of their constructor method.
Arguments not present in at construction time, will take default values from default settings
and user settings \textbf{replica-settings.cfg} files.

\begin{Verbatim}[commandchars=\\\{\}]
\PYG{g+go}{\PYGZgt{}\PYGZgt{}\PYGZgt{}from Replicas import Replica}
\PYG{g+go}{\PYGZgt{}\PYGZgt{}\PYGZgt{}replica = Replica(\PYGZsq{}customreplica\PYGZsq{}, nanos=40, temp=298)}
\PYG{g+go}{\PYGZgt{}\PYGZgt{}\PYGZgt{}print replica.nanos}
\PYG{g+go}{40}
\PYG{g+go}{\PYGZgt{}\PYGZgt{}\PYGZgt{}print replica.temp}
\PYG{g+go}{298}
\PYG{g+go}{\PYGZgt{}\PYGZgt{}\PYGZgt{}defreplica = Replica()}
\PYG{g+go}{\PYGZgt{}\PYGZgt{}\PYGZgt{}print defreplica.nanos}
\PYG{g+go}{20}
\PYG{g+go}{\PYGZgt{}\PYGZgt{}\PYGZgt{}print defreplica.temp}
\PYG{g+go}{300}
\end{Verbatim}

\emph{temp} and \emph{nanos} attributes where assigned from user configuration file which should be at user's home
directory \$HOME/.mdmix/replica-settings.cfg. Values not explicitely assigned there, will be taken from
default configurations in package installation directory.

This system for building instances permits the developer to modify/add/remove attributes to the instance without
modifying any code. For instance, if a new pair \emph{int-myattr=200} is written in user's replica configuration file,
default replicas will also have that new attribute.


\section{Replica class}
\label{replicas:replica-class}\index{Replica (class in pyMDMix.Replicas)}

\begin{fulllineitems}
\phantomsection\label{replicas:pyMDMix.Replicas.Replica}\pysiglinewithargsret{\strong{class }\code{pyMDMix.Replicas.}\bfcode{Replica}}{\emph{solvent=None}, \emph{name=None}, \emph{pdb=None}, \emph{crd=None}, \emph{top=None}, \emph{path=None}, \emph{extraResidues=}\optional{}, \emph{restrMask='`}, \emph{alignMask='`}, \emph{refPdb=None}, \emph{**kwargs}}{}
Class to contain an independent simulation run (a Replica).
Create folders, input files and control completeness of the different steps.
\index{\_\_init\_\_() (pyMDMix.Replicas.Replica method)}

\begin{fulllineitems}
\phantomsection\label{replicas:pyMDMix.Replicas.Replica.__init__}\pysiglinewithargsret{\bfcode{\_\_init\_\_}}{\emph{solvent=None}, \emph{name=None}, \emph{pdb=None}, \emph{crd=None}, \emph{top=None}, \emph{path=None}, \emph{extraResidues=}\optional{}, \emph{restrMask='`}, \emph{alignMask='`}, \emph{refPdb=None}, \emph{**kwargs}}{}
Constructor method for Replica objects.
\begin{quote}\begin{description}
\item[{Parameters}] \leavevmode\begin{itemize}
\item {} 
\textbf{name} (\emph{str}) -- Replica name. Name should be given now or later with setName(name).

\item {} 
\textbf{solvent} (\emph{str}) -- Solvent name. It must exist in Solvent database.

\item {} 
\textbf{pdb} (\emph{str}) -- PDB file containing the solvated system. To be created if still non existant.

\item {} 
\textbf{crd} (\emph{str}) -- Amber PARMCRD file of the solvated system. To be created if still non existant.

\item {} 
\textbf{top} (\emph{str}) -- Amber PRMTOP file. To be created if still non existant.

\item {} 
\textbf{path} (\emph{str}) -- Path to replica folder structure. If not given now, can be assigned with {\hyperref[replicas:pyMDMix.Replicas.Replica.setPath]{\code{setPath()}}}.

\item {} 
\textbf{extraResidues} (\emph{list}) -- List of non-standard residue names we wish to consider as part of the `solute' (important with auto mask detection).

\item {} 
\textbf{restrMask} (\emph{str}) -- Amber format mask to select residues and atoms to be restrained (if needed). If emtpy and restrains are requested, an automatic mask will be calculated from the pdb

\item {} 
\textbf{alignMask} (\emph{str}) -- Amber format mask to select atoms and residues over which trajectory should be aligned. If empty, an automatic mask will be calcualted.

\item {} 
\textbf{refPdb} (\emph{str}) -- Path to a PDB file used as reference for trajectory alignment.

\end{itemize}

\item[{Keywords}] \leavevmode
This provides a very flexible attribute assignment system.
- Every pair key=value will be assigned as an attribute to current replica.
- Pair values not given will take default values from Global Settings or User Settings specifications.

\end{description}\end{quote}

\end{fulllineitems}

\index{addFF() (pyMDMix.Replicas.Replica method)}

\begin{fulllineitems}
\phantomsection\label{replicas:pyMDMix.Replicas.Replica.addFF}\pysiglinewithargsret{\bfcode{addFF}}{\emph{ffname}}{}
Add forcefield parameters or FF modification files to be loaded when preparing the system with tLeap.

\end{fulllineitems}

\index{attach() (pyMDMix.Replicas.Replica method)}

\begin{fulllineitems}
\phantomsection\label{replicas:pyMDMix.Replicas.Replica.attach}\pysiglinewithargsret{\bfcode{attach}}{\emph{object}, \emph{attachname}, \emph{desc='`}}{}
Attach an object to this replica. This will create a pickle of the object
with a temporary filename and store the pickle file name with a \emph{name} and \emph{description}
in current replica attribute \code{Replica.attached} dictionary.
\begin{quote}\begin{description}
\item[{Parameters}] \leavevmode\begin{itemize}
\item {} 
\textbf{object} (\emph{any}) -- Object to pickle and link to the replica

\item {} 
\textbf{attachname} (\emph{str}) -- Name to assign to the attachment

\item {} 
\textbf{desc} (\emph{str}) -- Description. Optional.

\end{itemize}

\end{description}\end{quote}

\end{fulllineitems}

\index{createFolders() (pyMDMix.Replicas.Replica method)}

\begin{fulllineitems}
\phantomsection\label{replicas:pyMDMix.Replicas.Replica.createFolders}\pysiglinewithargsret{\bfcode{createFolders}}{}{}
Create directory tree for current replica. \code{path} should have been set
with {\hyperref[replicas:pyMDMix.Replicas.Replica.setPath]{\code{setPath()}}}. Copy inside the top/crd/pdb files if given.
\begin{description}
\item[{Tree structure:}] \leavevmode\begin{description}
\item[{replica.name/}] \leavevmode
replica.minfolder/
replica.eqfolder/
replica.mdfolder/

\end{description}

\end{description}

\end{fulllineitems}

\index{createMDInput() (pyMDMix.Replicas.Replica method)}

\begin{fulllineitems}
\phantomsection\label{replicas:pyMDMix.Replicas.Replica.createMDInput}\pysiglinewithargsret{\bfcode{createMDInput}}{}{}
Create MD input config files for the program selected (AMBER or NAMD).

\end{fulllineitems}

\index{createSystemFromOFF() (pyMDMix.Replicas.Replica method)}

\begin{fulllineitems}
\phantomsection\label{replicas:pyMDMix.Replicas.Replica.createSystemFromOFF}\pysiglinewithargsret{\bfcode{createSystemFromOFF}}{\emph{systemoff}, \emph{systemunit}, \emph{prmtop=None}, \emph{prmcrd=None}, \emph{pdb=None}}{}
Using tLeap, create a AMBER PRMTOP and PRMCRD from the \emph{systemoff} filepath and unitname \emph{systemunit}.
Solvent box will be added according to the solvent name used when instantiating the Replica.
Files will be saved inside replica folder.
\begin{quote}\begin{description}
\item[{Parameters}] \leavevmode\begin{itemize}
\item {} 
\textbf{systemoff} (\emph{str}) -- path to Amber Object File with system saved.

\item {} 
\textbf{systemunit} (\emph{str}) -- name of the unit saved inside \emph{systemoff} to be prepared.

\item {} 
\textbf{prmtop} (\emph{str}) -- name of the PARMTOP file to be saved. Default name will be constructed from \emph{systemunit} and name of replica. E.g.: MYSYS\_ETA0.prmtop

\item {} 
\textbf{prmcrd} (\emph{str}) -- name of the PARMCRD file to be saved. Default name will be constructed as for PARMTOP with extension .parmcrd

\item {} 
\textbf{pdb} (\emph{str}) -- name of the PDB file to be saved from the PARMTOP and PARMCRD files. Default name constructed as the otehrs with extension .pdb

\end{itemize}

\end{description}\end{quote}

\end{fulllineitems}

\index{dettach() (pyMDMix.Replicas.Replica method)}

\begin{fulllineitems}
\phantomsection\label{replicas:pyMDMix.Replicas.Replica.dettach}\pysiglinewithargsret{\bfcode{dettach}}{\emph{attachname}}{}
Remove attachement with name \textbf{attachname}

\end{fulllineitems}

\index{folderscreated() (pyMDMix.Replicas.Replica method)}

\begin{fulllineitems}
\phantomsection\label{replicas:pyMDMix.Replicas.Replica.folderscreated}\pysiglinewithargsret{\bfcode{folderscreated}}{}{}
Return \textbf{True} if replica directory structure is created.

\end{fulllineitems}

\index{getAttached() (pyMDMix.Replicas.Replica method)}

\begin{fulllineitems}
\phantomsection\label{replicas:pyMDMix.Replicas.Replica.getAttached}\pysiglinewithargsret{\bfcode{getAttached}}{\emph{attachname}}{}
Load attached object. See ::method::\emph{attach} for more info.
\begin{quote}\begin{description}
\item[{Returns}] \leavevmode
Unpickled object.

\end{description}\end{quote}

\end{fulllineitems}

\index{go() (pyMDMix.Replicas.Replica method)}

\begin{fulllineitems}
\phantomsection\label{replicas:pyMDMix.Replicas.Replica.go}\pysiglinewithargsret{\bfcode{go}}{}{}
Move to replica folder if created

\end{fulllineitems}

\index{importData() (pyMDMix.Replicas.Replica method)}

\begin{fulllineitems}
\phantomsection\label{replicas:pyMDMix.Replicas.Replica.importData}\pysiglinewithargsret{\bfcode{importData}}{\emph{**kwargs}}{}
Import existing data into current replica. Useful when analyzing data from external simulations
not run under pyMDMix.
\begin{quote}\begin{description}
\item[{Keywords}] \leavevmode
Give key=value pairs to be imported where:
- \textbf{value}: absolute path of existing folder containing data to link or a existing file.
- \textbf{key}: repica folder or file where to link data to:
\begin{quote}
\begin{description}
\item[{FILES:}] \leavevmode\begin{itemize}
\item {} 
\emph{pdb}:       System PDB

\item {} 
\emph{top}:       System Amber Topology

\item {} 
\emph{crd}:       System Amber Coordinates

\item {} 
\emph{solvent}:   Simulated solvent

\end{itemize}

\item[{FOLDERS}] \leavevmode\begin{itemize}
\item {} 
\emph{mdfolder}:      Production trajectory and output files

\item {} 
\emph{eqfolder}:      Equilibration folder

\item {} 
\emph{alignfolder}:   Aligned trajectory folder

\item {} 
\emph{densityfolder}: Containing density grids

\item {} 
\emph{energyfolder}:  Contraining energy converted grids

\end{itemize}

\end{description}
\end{quote}

All keys are optional. Only keys assigned will be imported.
\begin{description}
\item[{Example::\textasciitilde{}}] \leavevmode
\begin{Verbatim}[commandchars=\\\{\}]
\PYG{g+gp}{\PYGZgt{}\PYGZgt{}\PYGZgt{} }\PYG{n}{replica} \PYG{o}{=} \PYG{n}{Replica}\PYG{p}{(}\PYG{l+s}{\PYGZsq{}}\PYG{l+s}{ETA}\PYG{l+s}{\PYGZsq{}}\PYG{p}{,} \PYG{n}{name}\PYG{o}{=}\PYG{l+s}{\PYGZsq{}}\PYG{l+s}{test}\PYG{l+s}{\PYGZsq{}}\PYG{p}{)}
\PYG{g+gp}{\PYGZgt{}\PYGZgt{}\PYGZgt{} }\PYG{n}{replica}\PYG{o}{.}\PYG{n}{importData}\PYG{p}{(}\PYG{n}{pdb}\PYG{o}{=}\PYG{l+s}{\PYGZsq{}}\PYG{l+s}{/oldfolder/system.pdb}\PYG{l+s}{\PYGZsq{}}\PYG{p}{,} \PYG{n}{crd}\PYG{o}{=}\PYG{l+s}{\PYGZsq{}}\PYG{l+s}{/oldpath/system.crd}\PYG{l+s}{\PYGZsq{}}\PYG{p}{,} \PYG{n}{top}\PYG{o}{=}\PYG{l+s}{\PYGZsq{}}\PYG{l+s}{/oldpath/system.top}\PYG{l+s}{\PYGZsq{}}\PYG{p}{,}
\PYG{g+gp}{\PYGZgt{}\PYGZgt{}\PYGZgt{} }                   \PYG{n}{mdfolder}\PYG{o}{=}\PYG{l+s}{\PYGZsq{}}\PYG{l+s}{/oldpath/production}\PYG{l+s}{\PYGZsq{}}\PYG{p}{,} \PYG{n}{eqfolder}\PYG{o}{=}\PYG{l+s}{\PYGZsq{}}\PYG{l+s}{/oldpath/equilibration}\PYG{l+s}{\PYGZsq{}}\PYG{p}{)}
\end{Verbatim}

\end{description}

E.g. /oldfolder/production content will be linked into \code{pyMDMix.Replicas.Replica.mdfolder} folder.

\end{description}\end{quote}

\end{fulllineitems}

\index{iscreated() (pyMDMix.Replicas.Replica method)}

\begin{fulllineitems}
\phantomsection\label{replicas:pyMDMix.Replicas.Replica.iscreated}\pysiglinewithargsret{\bfcode{iscreated}}{}{}
Return \textbf{True} if replica folder and MD inputs have been written

\end{fulllineitems}

\index{mdinputwritten() (pyMDMix.Replicas.Replica method)}

\begin{fulllineitems}
\phantomsection\label{replicas:pyMDMix.Replicas.Replica.mdinputwritten}\pysiglinewithargsret{\bfcode{mdinputwritten}}{}{}
Return \textbf{True} if replica MD input files are writen.

\end{fulllineitems}

\index{setName() (pyMDMix.Replicas.Replica method)}

\begin{fulllineitems}
\phantomsection\label{replicas:pyMDMix.Replicas.Replica.setName}\pysiglinewithargsret{\bfcode{setName}}{\emph{name}}{}
Set replica name. Adapt logger.

\end{fulllineitems}

\index{setNanos() (pyMDMix.Replicas.Replica method)}

\begin{fulllineitems}
\phantomsection\label{replicas:pyMDMix.Replicas.Replica.setNanos}\pysiglinewithargsret{\bfcode{setNanos}}{\emph{nanos}}{}
Change number of nanoseconds for current replica

\end{fulllineitems}

\index{setOutFileTemplate() (pyMDMix.Replicas.Replica method)}

\begin{fulllineitems}
\phantomsection\label{replicas:pyMDMix.Replicas.Replica.setOutFileTemplate}\pysiglinewithargsret{\bfcode{setOutFileTemplate}}{\emph{outfiletemplate}}{}
Set/Modify output filename template for current replica. All filename templates must include \{nano\} and \{extension\}.
E.g.: md\{nano\}.\{extension\}

\end{fulllineitems}

\index{setPath() (pyMDMix.Replicas.Replica method)}

\begin{fulllineitems}
\phantomsection\label{replicas:pyMDMix.Replicas.Replica.setPath}\pysiglinewithargsret{\bfcode{setPath}}{\emph{path}, \emph{update=True}}{}
Set replica path. If update is True, update subfolder paths.

\end{fulllineitems}

\index{updateFromSettings() (pyMDMix.Replicas.Replica method)}

\begin{fulllineitems}
\phantomsection\label{replicas:pyMDMix.Replicas.Replica.updateFromSettings}\pysiglinewithargsret{\bfcode{updateFromSettings}}{}{}
Auxiliary function to update object with attributes from configuration files.

\end{fulllineitems}

\index{updatePaths() (pyMDMix.Replicas.Replica method)}

\begin{fulllineitems}
\phantomsection\label{replicas:pyMDMix.Replicas.Replica.updatePaths}\pysiglinewithargsret{\bfcode{updatePaths}}{}{}
Update replicaPaths using \code{path} as base path

\end{fulllineitems}


\end{fulllineitems}



\section{Examples}
\label{replicas:examples}

\subsection{Importing existing data}
\label{replicas:importing-existing-data}
In this example we will create an empty replica folder and add existing
sources from an imaginary previous simulation. It was run with ethanol mixture (named \emph{ETA}) for 40ns.

\begin{Verbatim}[commandchars=\\\{\}]
\PYG{g+gp}{\PYGZgt{}\PYGZgt{}\PYGZgt{} }\PYG{n}{previousdata} \PYG{o}{=} \PYG{p}{\PYGZob{}}\PYG{l+s}{\PYGZsq{}}\PYG{l+s}{pdb}\PYG{l+s}{\PYGZsq{}}\PYG{p}{:}\PYG{l+s}{\PYGZsq{}}\PYG{l+s}{/some/path/system.pdb}\PYG{l+s}{\PYGZsq{}}\PYG{p}{,} \PYG{l+s}{\PYGZsq{}}\PYG{l+s}{crd}\PYG{l+s}{\PYGZsq{}}\PYG{p}{:}\PYG{l+s}{\PYGZsq{}}\PYG{l+s}{/some/path/system.crd}\PYG{l+s}{\PYGZsq{}}\PYG{p}{,}\PYG{l+s}{\PYGZsq{}}\PYG{l+s}{top}\PYG{l+s}{\PYGZsq{}}\PYG{p}{:}\PYG{l+s}{\PYGZsq{}}\PYG{l+s}{/some/path/system.top}\PYG{l+s}{\PYGZsq{}}\PYG{p}{,}\PYG{l+s}{\PYGZsq{}}\PYG{l+s}{mdfolder}\PYG{l+s}{\PYGZsq{}}\PYG{p}{:}\PYG{l+s}{\PYGZsq{}}\PYG{l+s}{/some/path/production}\PYG{l+s}{\PYGZsq{}}\PYG{p}{,}\PYG{l+s}{\PYGZsq{}}\PYG{l+s}{eqfolder}\PYG{l+s}{\PYGZsq{}}\PYG{p}{:}\PYG{l+s}{\PYGZsq{}}\PYG{l+s}{/some/path/equilibration}\PYG{l+s}{\PYGZsq{}}\PYG{p}{\PYGZcb{}}
\PYG{g+gp}{\PYGZgt{}\PYGZgt{}\PYGZgt{} }\PYG{k+kn}{from} \PYG{n+nn}{Replicas} \PYG{k+kn}{import} \PYG{n}{Replica}
\PYG{g+gp}{\PYGZgt{}\PYGZgt{}\PYGZgt{} }\PYG{n}{replica} \PYG{o}{=} \PYG{n}{Replica}\PYG{p}{(}\PYG{l+s}{\PYGZsq{}}\PYG{l+s}{mynewreplica}\PYG{l+s}{\PYGZsq{}}\PYG{p}{,} \PYG{n}{nanos}\PYG{o}{=}\PYG{l+m+mi}{40}\PYG{p}{,} \PYG{n}{solvent}\PYG{o}{=}\PYG{l+s}{\PYGZsq{}}\PYG{l+s}{ETA}\PYG{l+s}{\PYGZsq{}}\PYG{p}{)}
\PYG{g+gp}{\PYGZgt{}\PYGZgt{}\PYGZgt{} }\PYG{n}{replica}\PYG{o}{.}\PYG{n}{createFolders}\PYG{p}{(}\PYG{p}{)}
\PYG{g+gp}{\PYGZgt{}\PYGZgt{}\PYGZgt{} }\PYG{n}{replica}\PYG{o}{.}\PYG{n}{importData}\PYG{p}{(}\PYG{o}{*}\PYG{o}{*}\PYG{n}{previousdata}\PYG{p}{)} \PYG{c}{\PYGZsh{} This will link all existing files inside the created folders}
\end{Verbatim}


\chapter{Solvents Module documentation}
\label{solvents::doc}\label{solvents:module-pyMDMix.Solvents}\label{solvents:solvents-module-documentation}\index{pyMDMix.Solvents (module)}
This module provides main {\hyperref[solvents:pyMDMix.Solvents.Solvent]{\code{Solvent}}} object class and
a {\hyperref[solvents:pyMDMix.Solvents.SolventManager]{\code{SolventManager}}} class to create/remove solvents
from the solvent database.

The easiest way to create a new solvent is through a configuration file
where all parameters can be assigned.

Second way is to instantiate the solvent directly giving all required
parameters to the constructor method.


\section{Instantiating and working with Solvent objects}
\label{solvents:instantiating-and-working-with-solvent-objects}
Solvent objects might be instantiated giving all required parameters to the constructor.
Here I exemplify it's usage and basic attributes:

\begin{Verbatim}[commandchars=\\\{\}]
\PYG{g+gp}{\PYGZgt{}\PYGZgt{}\PYGZgt{} }\PYG{k+kn}{import} \PYG{n+nn}{pyMDMix.tools} \PYG{k+kn}{as} \PYG{n+nn}{T}
\PYG{g+gp}{\PYGZgt{}\PYGZgt{}\PYGZgt{} }\PYG{k+kn}{import} \PYG{n+nn}{pyMDMix.Solvents} \PYG{k+kn}{as} \PYG{n+nn}{S}
\PYG{g+go}{\PYGZgt{}\PYGZgt{}\PYGZgt{}}
\PYG{g+gp}{\PYGZgt{}\PYGZgt{}\PYGZgt{} }\PYG{n}{off\PYGZus{}file} \PYG{o}{=} \PYG{n}{T}\PYG{o}{.}\PYG{n}{testRoot}\PYG{p}{(}\PYG{l+s}{\PYGZsq{}}\PYG{l+s}{ETAWAT20.off}\PYG{l+s}{\PYGZsq{}}\PYG{p}{)}   \PYG{c}{\PYGZsh{} Amber object file with box unit}
\PYG{g+gp}{\PYGZgt{}\PYGZgt{}\PYGZgt{} }\PYG{n}{name} \PYG{o}{=} \PYG{l+s}{\PYGZsq{}}\PYG{l+s}{ETA}\PYG{l+s}{\PYGZsq{}}
\PYG{g+gp}{\PYGZgt{}\PYGZgt{}\PYGZgt{} }\PYG{n}{probesmap} \PYG{o}{=} \PYG{p}{\PYGZob{}}\PYG{l+s}{\PYGZsq{}}\PYG{l+s}{OH}\PYG{l+s}{\PYGZsq{}}\PYG{p}{:}\PYG{l+s}{\PYGZsq{}}\PYG{l+s}{ETA@O1}\PYG{l+s}{\PYGZsq{}}\PYG{p}{,} \PYG{l+s}{\PYGZsq{}}\PYG{l+s}{CT}\PYG{l+s}{\PYGZsq{}}\PYG{p}{:}\PYG{l+s}{\PYGZsq{}}\PYG{l+s}{ETA@C2}\PYG{l+s}{\PYGZsq{}}\PYG{p}{\PYGZcb{}}   \PYG{c}{\PYGZsh{} Will link probe name OH with residue name ETA atom name O1}
\PYG{g+gp}{\PYGZgt{}\PYGZgt{}\PYGZgt{} }\PYG{n}{typesmap} \PYG{o}{=} \PYG{p}{\PYGZob{}}\PYG{l+s}{\PYGZsq{}}\PYG{l+s}{OH}\PYG{l+s}{\PYGZsq{}}\PYG{p}{:}\PYG{l+s}{\PYGZsq{}}\PYG{l+s}{Don,Acc}\PYG{l+s}{\PYGZsq{}}\PYG{p}{,} \PYG{l+s}{\PYGZsq{}}\PYG{l+s}{CT}\PYG{l+s}{\PYGZsq{}}\PYG{p}{:}\PYG{l+s}{\PYGZsq{}}\PYG{l+s}{Hyd}\PYG{l+s}{\PYGZsq{}}\PYG{p}{\PYGZcb{}}     \PYG{c}{\PYGZsh{} Link probe OH with types Donor and Acceptor. Link CT with hydrophobic. Names are arbitrary.}
\PYG{g+go}{\PYGZgt{}\PYGZgt{}\PYGZgt{}}
\PYG{g+gp}{\PYGZgt{}\PYGZgt{}\PYGZgt{} }\PYG{n}{boxunit} \PYG{o}{=} \PYG{l+s}{\PYGZsq{}}\PYG{l+s}{ETAWAT20}\PYG{l+s}{\PYGZsq{}} \PYG{c}{\PYGZsh{} As named inside the off file}
\PYG{g+gp}{\PYGZgt{}\PYGZgt{}\PYGZgt{} }\PYG{n}{info} \PYG{o}{=} \PYG{l+s}{\PYGZsq{}}\PYG{l+s}{Test direct instantiation of a Solvent object}\PYG{l+s}{\PYGZsq{}}
\PYG{g+go}{\PYGZgt{}\PYGZgt{}\PYGZgt{}}
\PYG{g+gp}{\PYGZgt{}\PYGZgt{}\PYGZgt{} }\PYG{c}{\PYGZsh{} Create instance}
\PYG{g+gp}{\PYGZgt{}\PYGZgt{}\PYGZgt{} }\PYG{n}{solv} \PYG{o}{=} \PYG{n}{S}\PYG{o}{.}\PYG{n}{Solvent}\PYG{p}{(}\PYG{n}{name}\PYG{o}{=}\PYG{n}{name}\PYG{p}{,} \PYG{n}{info}\PYG{o}{=}\PYG{n}{info}\PYG{p}{,} \PYG{n}{off}\PYG{o}{=}\PYG{n}{off\PYGZus{}file}\PYG{p}{,} \PYG{n}{probesmap}\PYG{o}{=}\PYG{n}{probesmap}\PYG{p}{,} \PYG{n}{typesmap}\PYG{o}{=}\PYG{n}{typesmap}\PYG{p}{,} \PYG{n}{boxunit}\PYG{o}{=}\PYG{n}{boxunit}\PYG{p}{)}
\PYG{g+gp}{\PYGZgt{}\PYGZgt{}\PYGZgt{} }\PYG{k}{print} \PYG{n}{solv}
\PYG{g+go}{SOLVENT: ETA}
\PYG{g+go}{INFO: Test direct instantiation of a Solvent object}
\PYG{g+go}{BOXUNIT: ETAWAT20}
\PYG{g+gp}{\PYGZgt{}\PYGZgt{}\PYGZgt{} }\PYG{k}{print} \PYG{n}{solv}\PYG{o}{.}\PYG{n}{probes} \PYG{c}{\PYGZsh{} print configured probes}
\PYG{g+go}{[CT, OH]}
\PYG{g+gp}{\PYGZgt{}\PYGZgt{}\PYGZgt{} }\PYG{k}{print} \PYG{n}{solv}\PYG{o}{.}\PYG{n}{types}  \PYG{c}{\PYGZsh{} set object}
\PYG{g+go}{set([\PYGZsq{}Acc\PYGZsq{}, \PYGZsq{}Don\PYGZsq{}, \PYGZsq{}Hyd\PYGZsq{}])}
\PYG{g+go}{\PYGZgt{}\PYGZgt{}\PYGZgt{}}
\PYG{g+gp}{\PYGZgt{}\PYGZgt{}\PYGZgt{} }\PYG{c}{\PYGZsh{} Calculate the probability of finding atom O1 of residue ETA in a grid voxel of 0.5x0.5x0.5 Angstroms}
\PYG{g+gp}{\PYGZgt{}\PYGZgt{}\PYGZgt{} }\PYG{k}{print} \PYG{n}{solv}\PYG{o}{.}\PYG{n}{getProbability}\PYG{p}{(}\PYG{l+s}{\PYGZsq{}}\PYG{l+s}{ETA}\PYG{l+s}{\PYGZsq{}}\PYG{p}{,}\PYG{l+s}{\PYGZsq{}}\PYG{l+s}{O1}\PYG{l+s}{\PYGZsq{}}\PYG{p}{,}\PYG{n}{voxel}\PYG{o}{=}\PYG{l+m+mf}{0.5}\PYG{o}{*}\PYG{o}{*}\PYG{l+m+mi}{3}\PYG{p}{)}
\PYG{g+go}{0.0005320194076762995}
\end{Verbatim}


\section{Adding new solvents to databases}
\label{solvents:adding-new-solvents-to-databases}
The Solvent object must be configured using {\hyperref[solvents:pyMDMix.Solvents.SolventManager.createSolvent]{\code{SolventManager.createSolvent()}}}
method giving a configuration file as argument.
Here is an exaple of valid configuraiton file with all available options commented:
\begin{quote}

\begin{Verbatim}[commandchars=\\\{\}]
\PYG{o}{[}GENERAL\PYG{o}{]}
\PYG{c}{\PYGZsh{} name to identify the mixture (ex: ION)}
\PYG{n+nv}{name} \PYG{o}{=} ETA 
\PYG{c}{\PYGZsh{} Any string to describe the box}
\PYG{n+nv}{info} \PYG{o}{=} Ethanol 20\PYGZpc{}\PYGZpc{} mixture
\PYG{c}{\PYGZsh{} path to off file containing the leap units}
\PYG{c}{\PYGZsh{} It should contain all parameters }
\PYG{n+nv}{objectfile} \PYG{o}{=} ETAWAT20.off 
\PYG{c}{\PYGZsh{} If the box contains waters, the name of the model (TIP3P, TIP4P, SCP..)}
\PYG{n+nv}{watermodel} \PYG{o}{=} TIP3P
\PYG{c}{\PYGZsh{} Name of the Leap box unit in object file(ex: IONWAT20)}
\PYG{n+nv}{boxunit} \PYG{o}{=} ETAWAT20

\PYG{o}{[}PROBES\PYG{o}{]}
\PYG{c}{\PYGZsh{} OPTIONAL SECTION}
\PYG{c}{\PYGZsh{} map probe names with residue@atoms (ie. NEG=COO@O1,O2)}
\PYG{c}{\PYGZsh{} probe names must be unique}
\PYG{n+nv}{WAT}\PYG{o}{=}WAT@O
\PYG{n+nv}{CT}\PYG{o}{=}ETA@C1
\PYG{n+nv}{OH}\PYG{o}{=}ETA@O1

\PYG{o}{[}TYPES\PYG{o}{]}
\PYG{c}{\PYGZsh{} OPTIONAL }
\PYG{c}{\PYGZsh{} Assign chemical types to the probes in previous section}
\PYG{c}{\PYGZsh{} Example: OH=Donor,Acceptor}
\PYG{n+nv}{OH}\PYG{o}{=}Don,Acc
\PYG{n+nv}{CT}\PYG{o}{=}Hyd
\PYG{n+nv}{WAT}\PYG{o}{=}Wat
\end{Verbatim}
\end{quote}

Configuring and adding a new solvent into the default database:

\begin{Verbatim}[commandchars=\\\{\}]
\PYG{g+gp}{\PYGZgt{}\PYGZgt{}\PYGZgt{} }\PYG{k+kn}{import} \PYG{n+nn}{tools} \PYG{k+kn}{as} \PYG{n+nn}{T}
\PYG{g+gp}{\PYGZgt{}\PYGZgt{}\PYGZgt{} }\PYG{k+kn}{from} \PYG{n+nn}{Solvents} \PYG{k+kn}{import} \PYG{n}{SolventManager}
\PYG{g+gp}{\PYGZgt{}\PYGZgt{}\PYGZgt{} }\PYG{n}{SM} \PYG{o}{=} \PYG{n}{SolventManager}\PYG{p}{(}\PYG{p}{)}
\PYG{g+go}{\PYGZgt{}\PYGZgt{}\PYGZgt{}}
\PYG{g+gp}{\PYGZgt{}\PYGZgt{}\PYGZgt{} }\PYG{c}{\PYGZsh{} Read configuration and create object.}
\PYG{g+gp}{\PYGZgt{}\PYGZgt{}\PYGZgt{} }\PYG{c}{\PYGZsh{} Object file path in the configuration file must be correct or}
\PYG{g+gp}{\PYGZgt{}\PYGZgt{}\PYGZgt{} }\PYG{c}{\PYGZsh{} errors will arise}
\PYG{g+gp}{\PYGZgt{}\PYGZgt{}\PYGZgt{} }\PYG{n}{configfile} \PYG{o}{=} \PYG{n}{T}\PYG{o}{.}\PYG{n}{testRoot}\PYG{p}{(}\PYG{l+s}{\PYGZsq{}}\PYG{l+s}{solvent\PYGZus{}template.cfg}\PYG{l+s}{\PYGZsq{}}\PYG{p}{)}
\PYG{g+gp}{\PYGZgt{}\PYGZgt{}\PYGZgt{} }\PYG{k}{print} \PYG{n}{configfile}
\PYG{g+go}{/Users/dalvarez/Dropbox/WORK/pyMDMix/pyMDMix/data/test/solvent\PYGZus{}template.cfg}
\PYG{g+gp}{\PYGZgt{}\PYGZgt{}\PYGZgt{} }\PYG{n}{newsolvent} \PYG{o}{=} \PYG{n}{SM}\PYG{o}{.}\PYG{n}{createSolvent}\PYG{p}{(}\PYG{n}{configfile}\PYG{p}{)}
\PYG{g+gp}{\PYGZgt{}\PYGZgt{}\PYGZgt{} }\PYG{k}{print} \PYG{n}{newsolvent}
\PYG{g+go}{SOLVENT: ETA}
\PYG{g+go}{INFO: Ethanol 20\PYGZpc{} mixture}
\PYG{g+go}{BOXUNIT: ETAWAT20}
\PYG{g+go}{\PYGZgt{}\PYGZgt{}\PYGZgt{}}
\PYG{g+gp}{\PYGZgt{}\PYGZgt{}\PYGZgt{} }\PYG{c}{\PYGZsh{} Add to the database}
\PYG{g+gp}{\PYGZgt{}\PYGZgt{}\PYGZgt{} }\PYG{n}{SM}\PYG{o}{.}\PYG{n}{saveSolvent}\PYG{p}{(}\PYG{n}{newsolvent}\PYG{p}{)}
\PYG{g+go}{ETA saved to database /Users/dalvarez/Dropbox/WORK/pyMDMix/pyMDMix/data/solventlib/SOLVENTS.db}
\PYG{g+gp}{\PYGZgt{}\PYGZgt{}\PYGZgt{} }\PYG{n}{SM}\PYG{o}{.}\PYG{n}{listSolvents}\PYG{p}{(}\PYG{p}{)}
\PYG{g+go}{[\PYGZsq{}PYZ\PYGZsq{}, \PYGZsq{}ISX\PYGZsq{}, \PYGZsq{}TFE\PYGZsq{}, \PYGZsq{}CLE\PYGZsq{}, \PYGZsq{}MSU\PYGZsq{}, \PYGZsq{}IMZ\PYGZsq{}, \PYGZsq{}ANT\PYGZsq{}, \PYGZsq{}ION\PYGZsq{}, \PYGZsq{}ETA\PYGZsq{}, \PYGZsq{}MOH\PYGZsq{}, \PYGZsq{}ISO5\PYGZsq{}, \PYGZsq{}ETAA\PYGZsq{}, \PYGZsq{}ISO\PYGZsq{}, \PYGZsq{}WAT\PYGZsq{}, \PYGZsq{}COM\PYGZsq{}, \PYGZsq{}PYR\PYGZsq{}, \PYGZsq{}MAM\PYGZsq{}]}
\end{Verbatim}

Configure and save object in a specific, empty database:

\begin{Verbatim}[commandchars=\\\{\}]
\PYG{g+gp}{\PYGZgt{}\PYGZgt{}\PYGZgt{} }\PYG{c}{\PYGZsh{} This action will copy all solvents in the default db to the new db and add the new solvent}
\PYG{g+gp}{\PYGZgt{}\PYGZgt{}\PYGZgt{} }\PYG{n}{customdb} \PYG{o}{=} \PYG{l+s}{\PYGZsq{}}\PYG{l+s}{mycustomdb.db}\PYG{l+s}{\PYGZsq{}}
\PYG{g+gp}{\PYGZgt{}\PYGZgt{}\PYGZgt{} }\PYG{n}{SM}\PYG{o}{.}\PYG{n}{saveSolvent}\PYG{p}{(}\PYG{n}{newsolvent}\PYG{p}{,} \PYG{n}{db}\PYG{o}{=}\PYG{n}{customdb}\PYG{p}{)}
\PYG{g+gp}{\PYGZgt{}\PYGZgt{}\PYGZgt{} }\PYG{n}{SM}\PYG{o}{.}\PYG{n}{listSolvents}\PYG{p}{(}\PYG{n}{customdb}\PYG{p}{)}
\PYG{g+go}{[\PYGZsq{}PYZ\PYGZsq{}, \PYGZsq{}ISX\PYGZsq{}, \PYGZsq{}TFE\PYGZsq{}, \PYGZsq{}CLE\PYGZsq{}, \PYGZsq{}MSU\PYGZsq{}, \PYGZsq{}IMZ\PYGZsq{}, \PYGZsq{}ANT\PYGZsq{}, \PYGZsq{}ION\PYGZsq{}, \PYGZsq{}ETA\PYGZsq{}, \PYGZsq{}MOH\PYGZsq{}, \PYGZsq{}ISO5\PYGZsq{}, \PYGZsq{}ETAA\PYGZsq{}, \PYGZsq{}ISO\PYGZsq{}, \PYGZsq{}WAT\PYGZsq{}, \PYGZsq{}COM\PYGZsq{}, \PYGZsq{}PYR\PYGZsq{}, \PYGZsq{}MAM\PYGZsq{}]}
\PYG{g+gp}{\PYGZgt{}\PYGZgt{}\PYGZgt{} }\PYG{c}{\PYGZsh{} Optionally, an empty database can be created and only add the new solvent}
\PYG{g+gp}{\PYGZgt{}\PYGZgt{}\PYGZgt{} }\PYG{n}{SM}\PYG{o}{.}\PYG{n}{saveSolvent}\PYG{p}{(}\PYG{n}{newsolvent}\PYG{p}{,} \PYG{n}{db}\PYG{o}{=}\PYG{l+s}{\PYGZsq{}}\PYG{l+s}{otherdb.db}\PYG{l+s}{\PYGZsq{}}\PYG{p}{,} \PYG{n}{createEmpty}\PYG{o}{=}\PYG{n+nb+bp}{True}\PYG{p}{)}
\PYG{g+gp}{\PYGZgt{}\PYGZgt{}\PYGZgt{} }\PYG{n}{SM}\PYG{o}{.}\PYG{n}{printSolvents}\PYG{p}{(}\PYG{l+s}{\PYGZsq{}}\PYG{l+s}{otherdb.db}\PYG{l+s}{\PYGZsq{}}\PYG{p}{)}
\PYG{g+go}{[\PYGZsq{}ETA\PYGZsq{}]}
\end{Verbatim}


\section{Solvent class}
\label{solvents:solvent-class}\index{Solvent (class in pyMDMix.Solvents)}

\begin{fulllineitems}
\phantomsection\label{solvents:pyMDMix.Solvents.Solvent}\pysiglinewithargsret{\strong{class }\code{pyMDMix.Solvents.}\bfcode{Solvent}}{\emph{name}, \emph{info}, \emph{offpath}, \emph{boxunit}, \emph{probesmap}, \emph{typesmap}, \emph{frcmodpaths=}\optional{}, \emph{watermodel='TIP3P'}, \emph{*args}, \emph{**kwargs}}{}
Solvent class for storing information on solvent mixtures for simulations
\index{\_\_init\_\_() (pyMDMix.Solvents.Solvent method)}

\begin{fulllineitems}
\phantomsection\label{solvents:pyMDMix.Solvents.Solvent.__init__}\pysiglinewithargsret{\bfcode{\_\_init\_\_}}{\emph{name}, \emph{info}, \emph{offpath}, \emph{boxunit}, \emph{probesmap}, \emph{typesmap}, \emph{frcmodpaths=}\optional{}, \emph{watermodel='TIP3P'}, \emph{*args}, \emph{**kwargs}}{}
Constructor of a Solvent object.
It expects some mandatory fields and some optional extra fields that can be assigned through kwargs.
\begin{quote}\begin{description}
\item[{Parameters}] \leavevmode\begin{itemize}
\item {} 
\textbf{name} (\emph{str}) -- Name identifying the current solvent mixture.

\item {} 
\textbf{info} (\emph{str}) -- Some string describing the solvent mixture. Will be used for printing the solvent.

\item {} 
\textbf{off} (\emph{str}) -- Filename that must exist. Will be used to fetch all information about the mixture and
also to later solvate the systems for simulation. Make sure this file is correct and the units
correctly working when setting up systems with it.

\item {} 
\textbf{boxunit} (\emph{str}) -- String specifying the name of the Leap unit containing the mixture inside the objectfile. Mandatory, specially important when more than 1 units are saved inside same object file.

\item {} 
\textbf{probesmap} (\emph{dict}) -- Map probe names to residues and atom names in the solvent box. Check documentation at the website or the documents for more information.

\item {} 
\textbf{typesmap} (\emph{dict}) -- Dictionary mapping chemical types to the probes in \emph{probesmap}.

\item {} 
\textbf{watermodel} (\emph{str}) -- If the solvent box contains waters, specify the water model used. Example; TIP3P, TIP4P... By default, it will be assigned to \code{S.DEF\_WATER\_BOX}

\end{itemize}

\item[{Keywords}] \leavevmode
key=value pairs that will be set as Solvent attributes. Useful for adding extra information in the solvent instance for easy access from custom functions.

\end{description}\end{quote}

\emph{probemap} and \emph{typesmap} Examples:

\begin{Verbatim}[commandchars=\\\{\}]
\PYG{g+gp}{\PYGZgt{}\PYGZgt{}\PYGZgt{} }\PYG{n}{probemap} \PYG{o}{=} \PYG{p}{\PYGZob{}}\PYG{l+s}{\PYGZsq{}}\PYG{l+s}{OH}\PYG{l+s}{\PYGZsq{}}\PYG{p}{:}\PYG{l+s}{\PYGZsq{}}\PYG{l+s}{ETA@O1}\PYG{l+s}{\PYGZsq{}}\PYG{p}{\PYGZcb{}} \PYG{c}{\PYGZsh{} Identify atom O1 of residue ETA as probe OH.}
\PYG{g+gp}{\PYGZgt{}\PYGZgt{}\PYGZgt{} }\PYG{n}{typesmap} \PYG{o}{=} \PYG{p}{\PYGZob{}}\PYG{l+s}{\PYGZsq{}}\PYG{l+s}{OH}\PYG{l+s}{\PYGZsq{}}\PYG{p}{:}\PYG{l+s}{\PYGZsq{}}\PYG{l+s}{Don,Acc}\PYG{l+s}{\PYGZsq{}}\PYG{p}{\PYGZcb{}} \PYG{c}{\PYGZsh{} Assign probe OH the chemical types Don and Acc (for Donor and Acceptor).}
\end{Verbatim}

\emph{kwargs} assignment example:

\begin{Verbatim}[commandchars=\\\{\}]
\PYG{g+gp}{\PYGZgt{}\PYGZgt{}\PYGZgt{} }\PYG{n}{solvent} \PYG{o}{=} \PYG{n}{Solvent}\PYG{p}{(}\PYG{n}{name}\PYG{o}{=}\PYG{l+s}{\PYGZsq{}}\PYG{l+s}{mysolvent}\PYG{l+s}{\PYGZsq{}}\PYG{p}{,}\PYG{n}{info}\PYG{o}{=}\PYG{l+s}{\PYGZsq{}}\PYG{l+s}{custom solvent}\PYG{l+s}{\PYGZsq{}}\PYG{p}{,}\PYG{n}{off}\PYG{o}{=}\PYG{l+s}{\PYGZsq{}}\PYG{l+s}{path/to/objectfile.off}\PYG{l+s}{\PYGZsq{}}\PYG{p}{,}\PYG{o}{.}\PYG{o}{.}\PYG{o}{.}\PYG{p}{,} \PYG{n}{myspecialattr} \PYG{o}{=} \PYG{l+m+mi}{300}\PYG{p}{)}
\PYG{g+gp}{\PYGZgt{}\PYGZgt{}\PYGZgt{} }\PYG{k}{print} \PYG{n}{solvent}\PYG{o}{.}\PYG{n}{myspecialattr}
\PYG{g+go}{300}
\end{Verbatim}

\end{fulllineitems}

\index{getProbability() (pyMDMix.Solvents.Solvent method)}

\begin{fulllineitems}
\phantomsection\label{solvents:pyMDMix.Solvents.Solvent.getProbability}\pysiglinewithargsret{\bfcode{getProbability}}{\emph{res}, \emph{atoms}, \emph{voxel=None}}{}
Obtain expected probability for all the atoms to fall into the voxel volume.
\begin{quote}\begin{description}
\item[{Parameters}] \leavevmode\begin{itemize}
\item {} 
\textbf{res} (\emph{str}) -- Residue name.

\item {} 
\textbf{atoms} (\emph{list}) -- Atom name list

\item {} 
\textbf{voxel} (\emph{float}) -- Volume of the voxel.
If not given, it is automatically calculated from defaults.

\end{itemize}

\item[{Returns}] \leavevmode
Probability of finding any of the atoms in a voxel.

\item[{Type}] \leavevmode
float

\end{description}\end{quote}

\end{fulllineitems}

\index{getProbeByName() (pyMDMix.Solvents.Solvent method)}

\begin{fulllineitems}
\phantomsection\label{solvents:pyMDMix.Solvents.Solvent.getProbeByName}\pysiglinewithargsret{\bfcode{getProbeByName}}{\emph{name}}{}
Returns {\hyperref[containers:pyMDMix.containers.Probe]{\code{Probe}}} instance with name \emph{name}. \textbf{False} otherwise.

\end{fulllineitems}

\index{getProbeProbability() (pyMDMix.Solvents.Solvent method)}

\begin{fulllineitems}
\phantomsection\label{solvents:pyMDMix.Solvents.Solvent.getProbeProbability}\pysiglinewithargsret{\bfcode{getProbeProbability}}{\emph{probename}}{}
Return probe probability

\end{fulllineitems}

\index{getProbesByType() (pyMDMix.Solvents.Solvent method)}

\begin{fulllineitems}
\phantomsection\label{solvents:pyMDMix.Solvents.Solvent.getProbesByType}\pysiglinewithargsret{\bfcode{getProbesByType}}{\emph{type}}{}
Returns a list of {\hyperref[containers:pyMDMix.containers.Probe]{\code{Probe}}} instances that match type \emph{type}.

\end{fulllineitems}

\index{getResidue() (pyMDMix.Solvents.Solvent method)}

\begin{fulllineitems}
\phantomsection\label{solvents:pyMDMix.Solvents.Solvent.getResidue}\pysiglinewithargsret{\bfcode{getResidue}}{\emph{resname}}{}
Return {\hyperref[containers:pyMDMix.containers.Residue]{\code{Residue}}} object with name \emph{resname}

\end{fulllineitems}

\index{isIonic() (pyMDMix.Solvents.Solvent method)}

\begin{fulllineitems}
\phantomsection\label{solvents:pyMDMix.Solvents.Solvent.isIonic}\pysiglinewithargsret{\bfcode{isIonic}}{}{}
Check wether the solvent box contains charged residues

\end{fulllineitems}

\index{writeOff() (pyMDMix.Solvents.Solvent method)}

\begin{fulllineitems}
\phantomsection\label{solvents:pyMDMix.Solvents.Solvent.writeOff}\pysiglinewithargsret{\bfcode{writeOff}}{\emph{name}}{}
Write object file of current solvent to filname \emph{name}

\end{fulllineitems}


\end{fulllineitems}



\section{SolventManager class}
\label{solvents:solventmanager-class}\index{SolventManager (class in pyMDMix.Solvents)}

\begin{fulllineitems}
\phantomsection\label{solvents:pyMDMix.Solvents.SolventManager}\pysigline{\strong{class }\code{pyMDMix.Solvents.}\bfcode{SolventManager}}
Class to manage solvent creation/removal and database manipulation
\index{createSolvent() (pyMDMix.Solvents.SolventManager method)}

\begin{fulllineitems}
\phantomsection\label{solvents:pyMDMix.Solvents.SolventManager.createSolvent}\pysiglinewithargsret{\bfcode{createSolvent}}{\emph{configfile}}{}
Create a Solvent isntance from the information in \emph{configfile} configuration file.
Examples of this file are available at template folder \code{T.templatesRoot()}
\begin{quote}\begin{description}
\item[{Parameters}] \leavevmode
\textbf{configfile} (\emph{str}) -- Filename with solvent configuration file

\item[{Return solvent}] \leavevmode
Return a Solvent object configured.

\item[{Return type}] \leavevmode
{\hyperref[solvents:pyMDMix.Solvents.Solvent]{\code{Solvent}}}

\end{description}\end{quote}

\end{fulllineitems}

\index{getDatabase() (pyMDMix.Solvents.SolventManager method)}

\begin{fulllineitems}
\phantomsection\label{solvents:pyMDMix.Solvents.SolventManager.getDatabase}\pysiglinewithargsret{\bfcode{getDatabase}}{\emph{db=None}}{}
Open the database / unpickle it.

\end{fulllineitems}

\index{getSolvent() (pyMDMix.Solvents.SolventManager method)}

\begin{fulllineitems}
\phantomsection\label{solvents:pyMDMix.Solvents.SolventManager.getSolvent}\pysiglinewithargsret{\bfcode{getSolvent}}{\emph{name}, \emph{db=None}}{}
Fetch solvent from the database by name.
\begin{quote}\begin{description}
\item[{Parameters}] \leavevmode\begin{itemize}
\item {} 
\textbf{name} (\emph{str}) -- Solvent name.

\item {} 
\textbf{db} (\emph{str}) -- Database path. If None, automatically detect.

\end{itemize}

\item[{Return}] \leavevmode
Solvent object from the database

\item[{Return type}] \leavevmode
{\hyperref[solvents:pyMDMix.Solvents.Solvent]{\code{Solvent}}} instance or False if not found.

\end{description}\end{quote}

\end{fulllineitems}

\index{listSolvents() (pyMDMix.Solvents.SolventManager method)}

\begin{fulllineitems}
\phantomsection\label{solvents:pyMDMix.Solvents.SolventManager.listSolvents}\pysiglinewithargsret{\bfcode{listSolvents}}{\emph{db=None}}{}
Fetch solvent names from the database.
\begin{quote}\begin{description}
\item[{Parameters}] \leavevmode
\textbf{db} (\emph{str}) -- Database path. If None, automatically detect.

\item[{Return}] \leavevmode
Name list

\item[{Return type}] \leavevmode
list

\end{description}\end{quote}

\end{fulllineitems}

\index{printSolvents() (pyMDMix.Solvents.SolventManager method)}

\begin{fulllineitems}
\phantomsection\label{solvents:pyMDMix.Solvents.SolventManager.printSolvents}\pysiglinewithargsret{\bfcode{printSolvents}}{\emph{db=None}}{}
Like list solvents but will print to screen information about the solvents.

\end{fulllineitems}

\index{removeSolvent() (pyMDMix.Solvents.SolventManager method)}

\begin{fulllineitems}
\phantomsection\label{solvents:pyMDMix.Solvents.SolventManager.removeSolvent}\pysiglinewithargsret{\bfcode{removeSolvent}}{\emph{solvName}, \emph{db=None}}{}
Remove solvent from database
Same as saveSolvent, \code{db} will be chosen automatically if None.
\begin{quote}\begin{description}
\item[{Parameters}] \leavevmode\begin{itemize}
\item {} 
\textbf{solvName} (\emph{str}) -- Solvent name.

\item {} 
\textbf{db} (\emph{str}) -- Database path. If None, automatically detect.

\end{itemize}

\item[{Raises SolventManagerError}] \leavevmode
if \emph{db} does not contain \emph{solvName}.

\end{description}\end{quote}

\end{fulllineitems}

\index{saveSolvent() (pyMDMix.Solvents.SolventManager method)}

\begin{fulllineitems}
\phantomsection\label{solvents:pyMDMix.Solvents.SolventManager.saveSolvent}\pysiglinewithargsret{\bfcode{saveSolvent}}{\emph{solvent}, \emph{db=None}, \emph{createEmpty=False}}{}
Save a Solvent isntance in the database \code{db} or default DB locations.
Selection of database is done in \code{self.\_\_getDatabase()}.
\begin{quote}\begin{description}
\item[{Parameters}] \leavevmode\begin{itemize}
\item {} 
\textbf{solvent} ({\hyperref[solvents:pyMDMix.Solvents.Solvent]{\code{Solvent}}}) -- Solvent object to save.

\item {} 
\textbf{db} (\emph{str}) -- Database where to save the solvent.
If None, default ones will be used (package DB if
the user can write there, or user DB otherwise).

\item {} 
\textbf{createEmpty} (\emph{bool}) -- If new database, create empty.
If False, copy data from package DB to the new DB.

\end{itemize}

\end{description}\end{quote}

\end{fulllineitems}


\end{fulllineitems}



\chapter{Containers Module documentation}
\label{containers:module-pyMDMix.containers}\label{containers::doc}\label{containers:containers-module-documentation}\index{pyMDMix.containers (module)}\index{Atom (class in pyMDMix.containers)}

\begin{fulllineitems}
\phantomsection\label{containers:pyMDMix.containers.Atom}\pysiglinewithargsret{\strong{class }\code{pyMDMix.containers.}\bfcode{Atom}}{\emph{id}, \emph{name}, \emph{type}, \emph{element}, \emph{charge}, \emph{*args}, \emph{**kwargs}}{}
Simple container for atomic information gathered in the OFF file: name, type, element, charge
\index{charge (pyMDMix.containers.Atom attribute)}

\begin{fulllineitems}
\phantomsection\label{containers:pyMDMix.containers.Atom.charge}\pysigline{\bfcode{charge}\strong{ = None}}
Partial charge

\end{fulllineitems}

\index{element (pyMDMix.containers.Atom attribute)}

\begin{fulllineitems}
\phantomsection\label{containers:pyMDMix.containers.Atom.element}\pysigline{\bfcode{element}\strong{ = None}}
Element. Integer.

\end{fulllineitems}

\index{id (pyMDMix.containers.Atom attribute)}

\begin{fulllineitems}
\phantomsection\label{containers:pyMDMix.containers.Atom.id}\pysigline{\bfcode{id}\strong{ = None}}
Integer. Atom ID.

\end{fulllineitems}

\index{name (pyMDMix.containers.Atom attribute)}

\begin{fulllineitems}
\phantomsection\label{containers:pyMDMix.containers.Atom.name}\pysigline{\bfcode{name}\strong{ = None}}
Atom name

\end{fulllineitems}

\index{type (pyMDMix.containers.Atom attribute)}

\begin{fulllineitems}
\phantomsection\label{containers:pyMDMix.containers.Atom.type}\pysigline{\bfcode{type}\strong{ = None}}
Atom AMBER TYPE

\end{fulllineitems}


\end{fulllineitems}

\index{Probe (class in pyMDMix.containers)}

\begin{fulllineitems}
\phantomsection\label{containers:pyMDMix.containers.Probe}\pysiglinewithargsret{\strong{class }\code{pyMDMix.containers.}\bfcode{Probe}}{\emph{name}, \emph{residue}, \emph{atoms}, \emph{type}, \emph{probability}}{}
Container for probe information. This object will store information about a particular probe linking
atom names, residues and chemical types with probabilities. Will also contain a mask for the residue.
\index{atoms (pyMDMix.containers.Probe attribute)}

\begin{fulllineitems}
\phantomsection\label{containers:pyMDMix.containers.Probe.atoms}\pysigline{\bfcode{atoms}\strong{ = None}}
Atom name list

\end{fulllineitems}

\index{name (pyMDMix.containers.Probe attribute)}

\begin{fulllineitems}
\phantomsection\label{containers:pyMDMix.containers.Probe.name}\pysigline{\bfcode{name}\strong{ = None}}
Name of the probe as given in \code{Solvent.probesmap}

\end{fulllineitems}

\index{p (pyMDMix.containers.Probe attribute)}

\begin{fulllineitems}
\phantomsection\label{containers:pyMDMix.containers.Probe.p}\pysigline{\bfcode{p}\strong{ = None}}
Probability of probe to be found in grid voxel volume

\end{fulllineitems}

\index{residue (pyMDMix.containers.Probe attribute)}

\begin{fulllineitems}
\phantomsection\label{containers:pyMDMix.containers.Probe.residue}\pysigline{\bfcode{residue}\strong{ = None}}
{\hyperref[containers:pyMDMix.containers.Residue]{\code{Residue}}} instance with corresponding residue information

\end{fulllineitems}

\index{type (pyMDMix.containers.Probe attribute)}

\begin{fulllineitems}
\phantomsection\label{containers:pyMDMix.containers.Probe.type}\pysigline{\bfcode{type}\strong{ = None}}
Chemical type

\end{fulllineitems}


\end{fulllineitems}

\index{Residue (class in pyMDMix.containers)}

\begin{fulllineitems}
\phantomsection\label{containers:pyMDMix.containers.Residue}\pysiglinewithargsret{\strong{class }\code{pyMDMix.containers.}\bfcode{Residue}}{\emph{name}, \emph{atoms}, \emph{connectivity}, \emph{xyz}, \emph{*args}, \emph{**kwargs}}{}
Simple container for whole residue unit information gathered in the OFF file.
Basically to sotre atomic information and possibly masks for later quick identify atom
positions.
\index{atids (pyMDMix.containers.Residue attribute)}

\begin{fulllineitems}
\phantomsection\label{containers:pyMDMix.containers.Residue.atids}\pysigline{\bfcode{atids}\strong{ = None}}
Dictionary mapping atom ids to {\hyperref[containers:pyMDMix.containers.Atom]{\code{Atom}}} instances

\end{fulllineitems}

\index{atnames (pyMDMix.containers.Residue attribute)}

\begin{fulllineitems}
\phantomsection\label{containers:pyMDMix.containers.Residue.atnames}\pysigline{\bfcode{atnames}\strong{ = None}}
Dictionary mapping atom names to {\hyperref[containers:pyMDMix.containers.Atom]{\code{Atom}}} instances

\end{fulllineitems}

\index{atoms (pyMDMix.containers.Residue attribute)}

\begin{fulllineitems}
\phantomsection\label{containers:pyMDMix.containers.Residue.atoms}\pysigline{\bfcode{atoms}\strong{ = None}}
List of {\hyperref[containers:pyMDMix.containers.Atom]{\code{Atom}}} instances that belong to residue

\end{fulllineitems}

\index{charge (pyMDMix.containers.Residue attribute)}

\begin{fulllineitems}
\phantomsection\label{containers:pyMDMix.containers.Residue.charge}\pysigline{\bfcode{charge}\strong{ = None}}
Total charge of the residue

\end{fulllineitems}

\index{connectivity (pyMDMix.containers.Residue attribute)}

\begin{fulllineitems}
\phantomsection\label{containers:pyMDMix.containers.Residue.connectivity}\pysigline{\bfcode{connectivity}\strong{ = None}}
Tuple with connectivity information

\end{fulllineitems}

\index{name (pyMDMix.containers.Residue attribute)}

\begin{fulllineitems}
\phantomsection\label{containers:pyMDMix.containers.Residue.name}\pysigline{\bfcode{name}\strong{ = None}}
Name of the residue

\end{fulllineitems}

\index{xyz (pyMDMix.containers.Residue attribute)}

\begin{fulllineitems}
\phantomsection\label{containers:pyMDMix.containers.Residue.xyz}\pysigline{\bfcode{xyz}\strong{ = None}}
XYZ coordinates in a numpy array Nx3

\end{fulllineitems}


\end{fulllineitems}



\chapter{Object File Management Module documentation}
\label{OFFManager:module-pyMDMix.OFFManager}\label{OFFManager:object-file-management-module-documentation}\label{OFFManager::doc}\index{pyMDMix.OFFManager (module)}
This module provides a reader for Amber OFF file format.
\begin{description}
\item[{Example::}] \leavevmode
\begin{Verbatim}[commandchars=\\\{\}]
\PYG{g+gp}{\PYGZgt{}\PYGZgt{}\PYGZgt{} }\PYG{k+kn}{import} \PYG{n+nn}{os.path} \PYG{k+kn}{as} \PYG{n+nn}{osp}
\PYG{g+gp}{\PYGZgt{}\PYGZgt{}\PYGZgt{} }\PYG{k+kn}{import} \PYG{n+nn}{pyMDMix.tools} \PYG{k+kn}{as} \PYG{n+nn}{T}
\PYG{g+gp}{\PYGZgt{}\PYGZgt{}\PYGZgt{} }\PYG{k+kn}{import} \PYG{n+nn}{pyMDMix.OFFManager} \PYG{k+kn}{as} \PYG{n+nn}{O}
\PYG{g+go}{\PYGZgt{}\PYGZgt{}\PYGZgt{}}
\PYG{g+gp}{\PYGZgt{}\PYGZgt{}\PYGZgt{} }\PYG{n}{f\PYGZus{}in} \PYG{o}{=} \PYG{n}{osp}\PYG{o}{.}\PYG{n}{join}\PYG{p}{(}\PYG{n}{T}\PYG{o}{.}\PYG{n}{testRoot}\PYG{p}{(}\PYG{p}{)}\PYG{p}{,} \PYG{l+s}{\PYGZsq{}}\PYG{l+s}{ETAWAT20.off}\PYG{l+s}{\PYGZsq{}}\PYG{p}{)}
\PYG{g+gp}{\PYGZgt{}\PYGZgt{}\PYGZgt{} }\PYG{n}{m} \PYG{o}{=} \PYG{n}{O}\PYG{o}{.}\PYG{n}{OFFManager}\PYG{p}{(}\PYG{n}{offFile}\PYG{o}{=}\PYG{n}{f\PYGZus{}in}\PYG{p}{)}
\PYG{g+go}{\PYGZgt{}\PYGZgt{}\PYGZgt{}}
\PYG{g+gp}{\PYGZgt{}\PYGZgt{}\PYGZgt{} }\PYG{k}{print} \PYG{n}{m}\PYG{o}{.}\PYG{n}{getUnits}\PYG{p}{(}\PYG{p}{)} \PYG{c}{\PYGZsh{} Get unit names present in the OFF file}
\PYG{g+go}{[\PYGZsq{}ETA\PYGZsq{},\PYGZsq{}ETAWAT20\PYGZsq{},\PYGZsq{}WAT\PYGZsq{}]}
\PYG{g+gp}{\PYGZgt{}\PYGZgt{}\PYGZgt{} }\PYG{k}{print} \PYG{n}{m}\PYG{o}{.}\PYG{n}{getResidueList}\PYG{p}{(}\PYG{l+s}{\PYGZsq{}}\PYG{l+s}{ETAWAT20}\PYG{l+s}{\PYGZsq{}}\PYG{p}{,} \PYG{n}{unique}\PYG{o}{=}\PYG{n+nb+bp}{True}\PYG{p}{)} \PYG{c}{\PYGZsh{} Get residues inside unit \PYGZsq{}ETAWAT20\PYGZsq{}}
\PYG{g+go}{[\PYGZsq{}WAT\PYGZsq{},\PYGZsq{}ETA\PYGZsq{}]}
\PYG{g+gp}{\PYGZgt{}\PYGZgt{}\PYGZgt{} }\PYG{k}{print} \PYG{n}{m}\PYG{o}{.}\PYG{n}{getVolume}\PYG{p}{(}\PYG{l+s}{\PYGZsq{}}\PYG{l+s}{ETAWAT20}\PYG{l+s}{\PYGZsq{}}\PYG{p}{)} \PYG{c}{\PYGZsh{} Volume of the box}
\PYG{g+go}{7988.43038}
\PYG{g+gp}{\PYGZgt{}\PYGZgt{}\PYGZgt{} }\PYG{k}{print} \PYG{n}{m}\PYG{o}{.}\PYG{n}{getNumRes}\PYG{p}{(}\PYG{l+s}{\PYGZsq{}}\PYG{l+s}{ETAWAT20}\PYG{l+s}{\PYGZsq{}}\PYG{p}{,} \PYG{l+s}{\PYGZsq{}}\PYG{l+s}{ETA}\PYG{l+s}{\PYGZsq{}}\PYG{p}{)} \PYG{c}{\PYGZsh{} Number of \PYGZsq{}ETA\PYGZsq{} residues inside \PYGZsq{}ETAWAT20\PYGZsq{} unit.}
\PYG{g+go}{17}
\end{Verbatim}

\end{description}
\index{OFFManager (class in pyMDMix.OFFManager)}

\begin{fulllineitems}
\phantomsection\label{OFFManager:pyMDMix.OFFManager.OFFManager}\pysiglinewithargsret{\strong{class }\code{pyMDMix.OFFManager.}\bfcode{OFFManager}}{\emph{offFile=None}, \emph{offString=None}, \emph{*args}, \emph{**kwargs}}{}
Manage Amber OFF file types. Only for reading.
\index{getAtoms() (pyMDMix.OFFManager.OFFManager method)}

\begin{fulllineitems}
\phantomsection\label{OFFManager:pyMDMix.OFFManager.OFFManager.getAtoms}\pysiglinewithargsret{\bfcode{getAtoms}}{\emph{unit}, \emph{skipH=False}}{}
Fetch atomic information for the unit selected.
Will return a dictionary with atom names and types.
\begin{quote}\begin{description}
\item[{Parameters}] \leavevmode\begin{itemize}
\item {} 
\textbf{unit} (\emph{str}) -- Unit to search.

\item {} 
\textbf{skipH} (\emph{bool}) -- If atom is Hydrogen, skip it.

\end{itemize}

\item[{Returns}] \leavevmode
List of {\hyperref[containers:pyMDMix.containers.Atom]{\code{Atom}}} instances containing id, name, atomtype, element and charge information.

\item[{Return type}] \leavevmode
list

\end{description}\end{quote}

\end{fulllineitems}

\index{getBoxDimensions() (pyMDMix.OFFManager.OFFManager method)}

\begin{fulllineitems}
\phantomsection\label{OFFManager:pyMDMix.OFFManager.OFFManager.getBoxDimensions}\pysiglinewithargsret{\bfcode{getBoxDimensions}}{\emph{unit}}{}
Get box dimension information from the object file for \code{self.boxunit}

\end{fulllineitems}

\index{getConnectivity() (pyMDMix.OFFManager.OFFManager method)}

\begin{fulllineitems}
\phantomsection\label{OFFManager:pyMDMix.OFFManager.OFFManager.getConnectivity}\pysiglinewithargsret{\bfcode{getConnectivity}}{\emph{unit}}{}
Fetch connectivity table for unit selected.
This is section !entry.UNIT.unit.connectivity in off file.
\begin{quote}\begin{description}
\item[{Parameters}] \leavevmode
\textbf{unit} (\emph{str}) -- Unit name

\item[{Returns}] \leavevmode
List with bonded pair indexes verbosely. Example: ((1,2),(1,3),(3,1),(2,1)...)

\item[{Return type}] \leavevmode
list

\end{description}\end{quote}

\end{fulllineitems}

\index{getCoords() (pyMDMix.OFFManager.OFFManager method)}

\begin{fulllineitems}
\phantomsection\label{OFFManager:pyMDMix.OFFManager.OFFManager.getCoords}\pysiglinewithargsret{\bfcode{getCoords}}{\emph{unit}}{}
Fetch positions information for unit selected.
This is section !entry.UNIT.unit.positions in off file.
\begin{quote}\begin{description}
\item[{Parameters}] \leavevmode
\textbf{unit} (\emph{str}) -- Unit name

\item[{Returns}] \leavevmode
Coordinates of unit atoms.

\item[{Return type}] \leavevmode
\code{numpy.ndarray} of floats with size \emph{Nx3}

\end{description}\end{quote}

\end{fulllineitems}

\index{getNumRes() (pyMDMix.OFFManager.OFFManager method)}

\begin{fulllineitems}
\phantomsection\label{OFFManager:pyMDMix.OFFManager.OFFManager.getNumRes}\pysiglinewithargsret{\bfcode{getNumRes}}{\emph{unit}, \emph{residue}}{}
Count the number of residues with name \emph{residue} inside unit \emph{unit}

\end{fulllineitems}

\index{getResidue() (pyMDMix.OFFManager.OFFManager method)}

\begin{fulllineitems}
\phantomsection\label{OFFManager:pyMDMix.OFFManager.OFFManager.getResidue}\pysiglinewithargsret{\bfcode{getResidue}}{\emph{res}, \emph{skipH=False}}{}
Fetch residue in off and return a {\hyperref[containers:pyMDMix.containers.Residue]{\code{Residue}}} instance
aontaining also atomic information.
\begin{quote}\begin{description}
\item[{Parameters}] \leavevmode\begin{itemize}
\item {} 
\textbf{unit} (\emph{str}) -- Residue name which should correspond to a valid unit in off file.

\item {} 
\textbf{skipH} (\emph{bool}) -- Skip hydrogen atom information. Default:False.

\end{itemize}

\item[{Returns}] \leavevmode
Residue instance info.

\item[{Return type}] \leavevmode
{\hyperref[containers:pyMDMix.containers.Residue]{\code{Residue}}} or \textbf{False} if unit not found.

\end{description}\end{quote}

\end{fulllineitems}

\index{getResidueList() (pyMDMix.OFFManager.OFFManager method)}

\begin{fulllineitems}
\phantomsection\label{OFFManager:pyMDMix.OFFManager.OFFManager.getResidueList}\pysiglinewithargsret{\bfcode{getResidueList}}{\emph{unit}, \emph{unique=True}}{}
Get a list of residue names for the \emph{unit} chosen.
\begin{quote}\begin{description}
\item[{Parameters}] \leavevmode\begin{itemize}
\item {} 
\textbf{unit} (\emph{str}) -- Unitname to search.

\item {} 
\textbf{unique} (\emph{bool}) -- If True, return a list with unique names. If False, the complete list of names will
be returned.

\end{itemize}

\item[{Returns}] \leavevmode
List with residuenames inside unit \emph{unit}.

\end{description}\end{quote}

\end{fulllineitems}

\index{getUnits() (pyMDMix.OFFManager.OFFManager method)}

\begin{fulllineitems}
\phantomsection\label{OFFManager:pyMDMix.OFFManager.OFFManager.getUnits}\pysiglinewithargsret{\bfcode{getUnits}}{}{}
Return the list of units int he object file.

\end{fulllineitems}

\index{getVolume() (pyMDMix.OFFManager.OFFManager method)}

\begin{fulllineitems}
\phantomsection\label{OFFManager:pyMDMix.OFFManager.OFFManager.getVolume}\pysiglinewithargsret{\bfcode{getVolume}}{\emph{unit}}{}
Get volume information from the object file for \code{self.boxunit}

\end{fulllineitems}

\index{hasUnit() (pyMDMix.OFFManager.OFFManager method)}

\begin{fulllineitems}
\phantomsection\label{OFFManager:pyMDMix.OFFManager.OFFManager.hasUnit}\pysiglinewithargsret{\bfcode{hasUnit}}{\emph{unitname}}{}
Return True if the OFF file has the unit with name `unitname'.

\end{fulllineitems}

\index{isParameter() (pyMDMix.OFFManager.OFFManager method)}

\begin{fulllineitems}
\phantomsection\label{OFFManager:pyMDMix.OFFManager.OFFManager.isParameter}\pysiglinewithargsret{\bfcode{isParameter}}{\emph{unit}}{}
Check if unit \emph{unit} is a parameter unit inside OFF.
\begin{quote}\begin{description}
\item[{Parameters}] \leavevmode
\textbf{unit} (\emph{str}) -- Name of the unit

\item[{Returns}] \leavevmode
True if its a parameter unit. False otherwise.

\end{description}\end{quote}

\end{fulllineitems}

\index{readOffSection() (pyMDMix.OFFManager.OFFManager method)}

\begin{fulllineitems}
\phantomsection\label{OFFManager:pyMDMix.OFFManager.OFFManager.readOffSection}\pysiglinewithargsret{\bfcode{readOffSection}}{\emph{unit}, \emph{section}, \emph{with\_header=False}}{}
Parse the object file and read a whole section for the unit selected.
\begin{quote}\begin{description}
\item[{Parameters}] \leavevmode\begin{itemize}
\item {} 
\textbf{unit} (\emph{str}) -- Unit name to search.

\item {} 
\textbf{section} (\emph{str}) -- OFF file section name. Example: \emph{residues} section will correspond to ''!entry.UNITNAME.unit.residues ..''  part of the file.

\item {} 
\textbf{with\_header} (\emph{bool}) -- If True, return output with heading line of the section.

\end{itemize}

\item[{Returns}] \leavevmode
Content of the section unitl next `!entry' is found. Returned with our without the heading line depending
on the value of \emph{with\_header}

\item[{Return type}] \leavevmode
list of strings

\end{description}\end{quote}

\end{fulllineitems}


\end{fulllineitems}

\index{Test (class in pyMDMix.OFFManager)}

\begin{fulllineitems}
\phantomsection\label{OFFManager:pyMDMix.OFFManager.Test}\pysiglinewithargsret{\strong{class }\code{pyMDMix.OFFManager.}\bfcode{Test}}{\emph{methodName='runTest'}}{}
Test
\index{test\_OFFManager() (pyMDMix.OFFManager.Test method)}

\begin{fulllineitems}
\phantomsection\label{OFFManager:pyMDMix.OFFManager.Test.test_OFFManager}\pysiglinewithargsret{\bfcode{test\_OFFManager}}{}{}
OFFManager test

\end{fulllineitems}


\end{fulllineitems}



\chapter{Setting General Options and Attributes through Configuration files}
\label{settings::doc}\label{settings:setting-general-options-and-attributes-through-configuration-files}
Several classes automatically take arguments from default configuration files distributed along the package.
Once pyMDMix is started for the first time, it will also make a copy of these configuration files inside the user's home directory
for easy modification. If any parameter is modified by the user, it will have higher priority and the default one will be ignored.
For restoring initial file, just remove it from the user directory.
\begin{description}
\item[{Mainly, three configuration files govern the program:}] \leavevmode\begin{itemize}
\item {} 
General configuration (\textbf{settings.cfg})

\item {} 
Replica configuration (\textbf{replica-settings.cfg})

\item {} 
Project configuration (\textbf{project-settings.cfg})

\end{itemize}

\end{description}


\section{General configuration}
\label{settings:general-configuration}
This is the default file for configuring general and project options in pyMDMix.
It can be found at the package installation directory (\$INSTALLDIR) under \textbf{\$INSTALLDIR/data/defaults/settings.cfg}
or at user's home directory \textbf{./mdmix/settings.cfg}

\begin{Verbatim}[commandchars=\\\{\}]
\PYG{k}{[MD]}
\PYG{n+na}{list\PYGZhy{}AVAIL\PYGZus{}MDPROG} \PYG{o}{=} \PYG{l+s}{AMBER, NAMD \PYGZsh{} IMPLEMENTED SIMULATION PROGRAMS}
\PYG{n+na}{float\PYGZhy{}AMBER\PYGZus{}SOLVATE\PYGZus{}BUFFER} \PYG{o}{=} \PYG{l+s}{13 \PYGZsh{} Buffer for solvateOct command in tLeap}
\PYG{n+na}{DEF\PYGZus{}AMBER\PYGZus{}WATBOX} \PYG{o}{=} \PYG{l+s}{TIP3P	\PYGZsh{} Default water model to use}
\PYG{n+na}{DEF\PYGZus{}MDPROGRAM} \PYG{o}{=} \PYG{l+s}{AMBER		\PYGZsh{} From implemented programs, default to use}
\PYG{n+na}{int\PYGZhy{}DEF\PYGZus{}TRAJFREQUENCY} \PYG{o}{=} \PYG{l+s}{500 			\PYGZsh{} Trajectory ritting frequency  = 1000 snapshots per nanosecond = DEF\PYGZus{}NVT\PYGZus{}PRODUCTION\PYGZus{}NSTEPS / DEF\PYGZus{}TRAJFREQUENCY}
\PYG{n+na}{int\PYGZhy{}DEF\PYGZus{}MINSTEPS} \PYG{o}{=} \PYG{l+s}{5000				\PYGZsh{} Number of minimization steps to run}
\PYG{n+na}{int\PYGZhy{}DEF\PYGZus{}HEATING\PYGZus{}STEPS\PYGZus{}PER\PYGZus{}FILE} \PYG{o}{=} \PYG{l+s}{100000 	\PYGZsh{} Heating steps. 100.000 steps = 200ps}
\PYG{n+na}{int\PYGZhy{}DEF\PYGZus{}HEATING\PYGZus{}TEMP\PYGZus{}INI} \PYG{o}{=} \PYG{l+s}{100 			\PYGZsh{} Start heating at 100 K}
\PYG{n+na}{int\PYGZhy{}DEF\PYGZus{}NPT\PYGZus{}EQUILIBRATION\PYGZus{}NSTEPS} \PYG{o}{=} \PYG{l+s}{500000 	\PYGZsh{} 1ns equilibration at NPT}
\PYG{n+na}{int\PYGZhy{}DEF\PYGZus{}NVT\PYGZus{}PRODUCTION\PYGZus{}NSTEPS} \PYG{o}{=} \PYG{l+s}{500000 		\PYGZsh{} 1ns production files}
\PYG{n+na}{int\PYGZhy{}DEF\PYGZus{}NAMD\PYGZus{}HEATING\PYGZus{}TOTALSTEPS} \PYG{o}{=} \PYG{l+s}{500000 	\PYGZsh{} 1ns equilibration time to increase temperature from TEMP\PYGZus{}INI}
\PYG{n+na}{int\PYGZhy{}DEF\PYGZus{}AMBER\PYGZus{}NETCDF} \PYG{o}{=} \PYG{l+s}{1    			\PYGZsh{} Write trajectory in NETCDF format by default}
\PYG{n+na}{list\PYGZhy{}DEF\PYGZus{}FF} \PYG{o}{=} \PYG{l+s}{leaprc.ff99SB, leaprc.gaff	\PYGZsh{} Default forcefield files to load when opening tLeap}

\PYG{k}{[GENERAL]}
\PYG{c+c1}{\PYGZsh{}\PYGZsh{} The following options are only ckecked for their type}
\PYG{c+c1}{\PYGZsh{}\PYGZsh{} use type\PYGZhy{}name if you want to enforce type conversion on a parameter}
\PYG{c+c1}{\PYGZsh{}\PYGZsh{} example: }
\PYG{c+c1}{\PYGZsh{}\PYGZsh{} int\PYGZhy{}param1 = 10 creates a variable param1 of type int with value 10}
\PYG{c+c1}{\PYGZsh{}\PYGZsh{} By contrast, param1 = 10 gives a variable param1 of type str with value \PYGZsq{}10\PYGZsq{}}

\PYG{n+na}{int\PYGZhy{}testparam} \PYG{o}{=} \PYG{l+s}{42	\PYGZsh{}\PYGZsh{} used for test code}
\PYG{n+na}{list\PYGZhy{}GRIDTYPES} \PYG{o}{=} \PYG{l+s}{MDMIX\PYGZus{}DENS,MDMIX\PYGZus{}CORR,MDMIX\PYGZus{}RAW,MDMIX\PYGZus{}OTHER,MDMIX\PYGZus{}UNK, MDMIX\PYGZus{}PART\PYGZus{}DENS, MDMIX\PYGZus{}RAW\PYGZus{}AVG}
\PYG{n+na}{AVGOUTPATH} \PYG{o}{=} \PYG{l+s}{PROBE\PYGZus{}AVG}
\PYG{n+na}{AVGOUTPREFIX} \PYG{o}{=} \PYG{l+s}{avg\PYGZus{}}
\PYG{n+na}{float\PYGZhy{}GRID\PYGZus{}SPACING} \PYG{o}{=} \PYG{l+s}{0.5}
\PYG{n+na}{DEBUG}\PYG{o}{=}\PYG{l+s}{0                 \PYGZsh{} If zero, no extra debug information will be printed. Put 1 for extra info.}
\end{Verbatim}


\section{Replica configuration}
\label{settings:replica-configuration}
\begin{Verbatim}[commandchars=\\\{\}]
\PYG{k}{[GENERAL]}
\PYG{c+c1}{\PYGZsh{}\PYGZsh{} The following options are only ckecked for their type}
\PYG{c+c1}{\PYGZsh{}\PYGZsh{} use type\PYGZhy{}name if you want to enforce type conversion on a parameter}
\PYG{c+c1}{\PYGZsh{}\PYGZsh{} example:}
\PYG{c+c1}{\PYGZsh{}\PYGZsh{} int\PYGZhy{}param1 = 10 creates a variable param1 of type int with value 10}
\PYG{c+c1}{\PYGZsh{}\PYGZsh{} By contrast, param1 = 10 gives a variable param1 of type str with value \PYGZsq{}10\PYGZsq{}}
\PYG{c+c1}{\PYGZsh{}\PYGZsh{} Possible types: int, float, bool, list.}
\PYG{c+c1}{\PYGZsh{}\PYGZsh{} list type will chop the string by commas. Eg. list\PYGZhy{}ff=a,b will become a list ff=[\PYGZsq{}a\PYGZsq{},\PYGZsq{}b\PYGZsq{}]}

\PYG{c+c1}{\PYGZsh{} Set simulation options}
\PYG{n+na}{netcdf} \PYG{o}{=} \PYG{l+s}{1                  \PYGZsh{} 1 (write trajectory in nc format) or 0 (write in ascii format)}
\PYG{n+na}{restrMode} \PYG{o}{=} \PYG{l+s}{FREE            \PYGZsh{} Restraining scheme: FREE, HA (heavy atoms) or BB (backbone only)}
\PYG{n+na}{float\PYGZhy{}restrForce} \PYG{o}{=} \PYG{l+s}{0.0      \PYGZsh{} Restraining force if applicable. Default 0 kcal/mol.A\PYGZca{}2}
\PYG{n+na}{int\PYGZhy{}nanos} \PYG{o}{=} \PYG{l+s}{20              \PYGZsh{} Production length in nanoseconds. Default: 20ns}
\PYG{n+na}{float\PYGZhy{}temp} \PYG{o}{=} \PYG{l+s}{300            \PYGZsh{} Simulation temperature. Default = 300K}
\PYG{n+na}{mdProgram} \PYG{o}{=} \PYG{l+s}{AMBER           \PYGZsh{} Default simulation program. Options: AMBER or NAMD currently}
\PYG{n+na}{int\PYGZhy{}trajfrequency} \PYG{o}{=} \PYG{l+s}{500     \PYGZsh{} Trajectory writing frequency  = 1000 snapshots per nanosecond = int\PYGZhy{}production\PYGZus{}nsteps / int\PYGZhy{}trajfrequency}
\PYG{n+na}{int\PYGZhy{}minsteps} \PYG{o}{=} \PYG{l+s}{5000				\PYGZsh{} Number of minimization steps to run}
\PYG{n+na}{int\PYGZhy{}heating\PYGZus{}steps} \PYG{o}{=} \PYG{l+s}{100000                      \PYGZsh{} Heating steps for each file. 100.000 steps = 200ps}
\PYG{n+na}{float\PYGZhy{}parm\PYGZus{}heating\PYGZus{}tempi} \PYG{o}{=} \PYG{l+s}{100 			\PYGZsh{} Start heating at 100 K}
\PYG{n+na}{int\PYGZhy{}npt\PYGZus{}eq\PYGZus{}steps} \PYG{o}{=} \PYG{l+s}{500000                       \PYGZsh{} 1ns equilibration at NPT}
\PYG{n+na}{int\PYGZhy{}nvt\PYGZus{}prod\PYGZus{}steps} \PYG{o}{=} \PYG{l+s}{500000                     \PYGZsh{} 1ns production files}
\PYG{n+na}{int\PYGZhy{}namd\PYGZus{}heating\PYGZus{}steps} \PYG{o}{=} \PYG{l+s}{500000                 \PYGZsh{} 1ns equilibration total time to increase temperature from float\PYGZhy{}heating\PYGZus{}tempi to float\PYGZhy{}temp in NAMD}
\PYG{n+na}{list\PYGZhy{}FF} \PYG{o}{=} \PYG{l+s}{leaprc.ff99SB, leaprc.gaff            \PYGZsh{} Default forcefield files to load when opening tLeap}
\PYG{n+na}{float\PYGZhy{}amber\PYGZus{}solvate\PYGZus{}buffer} \PYG{o}{=} \PYG{l+s}{14                 \PYGZsh{} Buffer for solvateOct command in tLeap}

\PYG{c+c1}{\PYGZsh{} Set filepaths}
\PYG{n+na}{mdfolder} \PYG{o}{=} \PYG{l+s}{md               \PYGZsh{} Name for the production folder}
\PYG{n+na}{eqfolder} \PYG{o}{=} \PYG{l+s}{eq               \PYGZsh{} Name for equilibration folder}
\PYG{n+na}{minfolder} \PYG{o}{=} \PYG{l+s}{min             \PYGZsh{} Name for minimization folder}
\PYG{n+na}{alignfolder} \PYG{o}{=} \PYG{l+s}{align         \PYGZsh{} Name for folder containing aligned trajectory}
\PYG{n+na}{energyfolder} \PYG{o}{=} \PYG{l+s}{egrids       \PYGZsh{} Name for folder containing energy grids}
\PYG{n+na}{densityfolder} \PYG{o}{=} \PYG{l+s}{dgrids      \PYGZsh{} Folder containing density/occupancy grids}

\PYG{c+c1}{\PYGZsh{} Next option is a string to be converted to trajectory file names in production and equilibration files.}
\PYG{c+c1}{\PYGZsh{} Always include \PYGZob{}nano\PYGZcb{} and \PYGZob{}extension\PYGZcb{} keywords.}
\PYG{c+c1}{\PYGZsh{} Example:    md\PYGZob{}nano\PYGZcb{}.\PYGZob{}extension\PYGZcb{}    (DEFAULT)}
\PYG{c+c1}{\PYGZsh{} Will be:    md1.nc  if netcdf used and 1st nanosecond of production.}
\PYG{n+na}{outfiletemplate} \PYG{o}{=} \PYG{l+s}{md\PYGZob{}nano\PYGZcb{}.\PYGZob{}extension\PYGZcb{}}
\end{Verbatim}

\#Settings Module
\#===============
\#
\#.. automodule:: Settings
\#       :members:


\chapter{Indices and tables}
\label{index:indices-and-tables}\begin{itemize}
\item {} 
\emph{genindex}

\item {} 
\emph{modindex}

\item {} 
\emph{search}

\end{itemize}


\renewcommand{\indexname}{Python Module Index}
\begin{theindex}
\def\bigletter#1{{\Large\sffamily#1}\nopagebreak\vspace{1mm}}
\bigletter{p}
\item {\texttt{pyMDMix.containers}}, \pageref{containers:module-pyMDMix.containers}
\item {\texttt{pyMDMix.OFFManager}}, \pageref{OFFManager:module-pyMDMix.OFFManager}
\item {\texttt{pyMDMix.Project}}, \pageref{project:module-pyMDMix.Project}
\item {\texttt{pyMDMix.Replicas}}, \pageref{replicas:module-pyMDMix.Replicas}
\item {\texttt{pyMDMix.Solvents}}, \pageref{solvents:module-pyMDMix.Solvents}
\end{theindex}

\renewcommand{\indexname}{Index}
\printindex
\end{document}
